\chapter{Conclusions}
    \section{Summary and Conclusions}
        This thesis presented the development of a foot placement planning system for a hexapod robot to enable it to traverse uneven terrain containing height changes, sloped surfaces, and edges.
        A simulation model of the existing physical hexapod robot, and its environment, was created in \ac{mujoco}, an advanced physics engine with superior contact physics modelling.
        A real time local dense-mapping system was developed by using depth images obtained from an \ac{rgbd} camera mounted on the robot. The mapping  system also utilised pose estimations from ORB-SLAM3, a stereoscopic visual SLAM algorithm. ORB-SLAM3 was implemented using a third party library.
        A walkability scoring system was developed to further process the heightmap generated by the mapping system, resulting in a walkability map. The walkability map indicates where on the terrain the robot can and can not safely step.
        Next, a baseline motion control system that enables the robot to walk on flat terrain was developed. The baseline system was then further modified with a algorithm that adjusts the nominal foot positions from the baseline motion control system based on the score map. The resulting new foot positions are then adjusted both in the horizontal and vertical plane to a position on the terrain that is safe to step on, all while the robot maintains a level body. The robot's floor height reference was also updated based on the terrain map, this allowed the robot to adjust its body height appropriately to the average terrain height. Furthermore, a new foot trajectory generation system based on a flow function was developed to produce a trajectory based on the horizontal and vertical distance to the foot's destination.

        The mapping, scoring, and optimal foot placement systems were tested in simulation and on practical RGB-D images recorded by the physical hexapod robot. Simulation tests were performed to demonstrate that the integrated system can successfully control the hexapod robot to walk on flat terrain, staircase terrain, and uneven terrain consisting of "cobblestones"  with grooves between them. The simulation results show that the hexapod is able to traverse both flat and uneven terrain, is able to adjust both its horizontal and vertical foot placements in response to the terrain.

        % Throughout this thesis a system to map terrain and score terrain for walkability was designed and implemented. Then a baseline tripod gait walking system was implemented, this system generated foot positions that would allow the hexapod robot to blindly walk over flat terrain. The walkability was then used to adjust the nominal foot positions from the baseline walking system to the terrain, in both the horizontal and vertical planes.

        % Various tests were performed to validate the effectiveness of these systems in simulation and practically on the physical hexapod. The tests revealed that the mapping and scoring systems performed well, with possible improvements in the frequency of the pose estimates from the SLAM system to reduce noise and artifacts in the heightmap.

        % The walking tests, performed in simulation, showed that the system is capable of successfully navigating uneven terrain with large vertical displacements, such as stairs, as well as terrain with "islands" of areas that are suitable for foot placement, such as a "cobblestone" surface.

    \section{Future Work}
        Below is listed possible future tests, improvements, and additions to the system.

        \begin{enumerate}
            \item Although the mapping and scoring was implemented and practically tested on the physical hexapod, the motion control system was only tested in simulation on uneven terrain. Future work should include practical tests with the physical hexapod traversing uneven terrain such as steps and cobblestones.            
            \item The variations in the height estimate received from the ORB-SLAM3 system caused artifacts in the heightmap and the score map. Methods to improve the height estimate should be investigated.
            \item The foot arcs are currently only adjusted for the new foot placements, but are not adjusted to avoid obstacles in the terrain. Future work could look at detected obstacles close to the legs and adjusting he arcs to avoid collisions.
            \item The hexapod currently "freezes" if it cannot find suitable alternate foot placement positions for the given terrain, and the human operator must then manually change the commanded direction of motion. A short-term path planner should be investigated that can autonomously change the hexapod's planned path to circumvent the terrain that cannot be crossed.
            \item The mapping and scoring system currently does not identify hazardous terrain such as pools of water or loose earth. Terrain classification and image segmentation algorithms could be investigated to add a "terrain type" score to to the walkability score.

            
        \end{enumerate}
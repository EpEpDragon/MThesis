\chapter{Conclusions}

Throughout this thesis a system to map terrain and score terrain for walkability was designed and implemented. Then a baseline tripod gait walking system
was implemented, this system generated foot positions that would allow the hexapod robot to blindly walk over flat terrain. The walkability was then used
to adjust the nominal foot positions from the baseline walking system to the terrain, in both the horizontal and vertical planes.

Various tests were performed to validate the effectiveness of these systems in simulation and in hardware. The tests revealed that the mapping and scoring
systems performed well, with possible improvements in the frequency of pose updates from the SLAM system to reduce noise and artifacts in the heightmap.

The walking testes, performed in simulation, showed that the system is capable of successfully navigating uneven terrain with large vertical 
displacements, such as stairs, and horizontal undulations, such as an assortment of rocks.

As the system focuses exclusively on the mapping and feet motion planning aspect of the robot, future work could include macro scale path planning and
waypoint navigation. Furthermore, terrain could be classified by not only its geometric shape, but also by its material characteristics, such as 
differentiating between sand, rock, dirt, etc.

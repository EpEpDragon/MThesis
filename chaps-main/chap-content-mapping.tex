\chapter{Mapping} \label{chap:mapping}
For accurate foot placement and localisation purposes the robot makes use of two maps, a sparse map covering a large area, and a dense map covering a small
area around the robot. The primarily use of the sparse map is for localisation and extracting pose data, i.e. orientation, velocity and rate. While the dense
map is used to analyse the terrain and find an appropriate point to place the three supporting feet.
It is possible to also use the sparse map for autonomous navigation, however this use case in not covered in this paper.
This chapter covers the design of the mapping system.

\section{Localisation And Sparse Map}
    The localisation, sparse mapping and pose estimation is handle by ORB-SLAM3 as described in \cite{campos2021orb}. Since ORB-SLAM3 is not a system designed by the author, its
    design will not be covered in this chapter. Implementation and operation details will however be covered in chapters \ref{chap:hardware} and \ref{chap:results}.

\section{Dense Map}

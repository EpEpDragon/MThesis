\chapter{End Effector Anchor Point Adjustment} \label{chap:effector-placement}
The best possible anchor points for the supporting feet, given an initial point from the walking
gait state machine, must be found based on the heightmap. This is done by applying various
scores to the heightmap and selecting a point with the best score within an allowable
deviation from the initial point received from the gait state machine.

\section{Scoring}
    The scores considered are the gradient and proximity scores. The gradient score aims to reject points with high gradients while the proximity score rejects points close
    to other parts of the terrain with steep inclines, i.e. to reject points inside holes or close to ledges.
    \subsection{Gradient Score}
        The gradient score is simply taken as the slope of the terrain at the current point. The aim of this score is to prevent the robot from slipping due to
        selecting anchor points with too steep of a gradient.
        
        As the heightmap slope is not defined by a known function, the gradient is calculated using the Sobel
        operator (\cite{sobel2014}), a combination of a central finite difference and a smoothing operator.
        
        Equation \ref{eq:sobelgx} and \ref{eq:sobelgy} describe the two separable x and y direction kernels, \(G_x\) and \(G_y\). These kernels are a combination of
        central finite difference and smoothing operator, see Appendix \ref{app:sobel} for a breakdown. Equation \ref{eq:sobelsg} combines \(G_x\) and \(G_y\)
        to produce the gradient score, \(S_g\).

        \begin{align}
            \begin{split} \label{eq:sobelgx}
                G_x(x,y) &=  \begin{bmatrix}
                                +1 & 0 & -1\\
                                +2 & 0 & -2\\
                                +1 & 0 & -1
                            \end{bmatrix}
                            *\boldsymbol{h}_{i,j}
            \end{split}\\
            \begin{split} \label{eq:sobelgy}
                G_y(x,y) &=  \begin{bmatrix}
                                +1 & +2 & +1\\
                                0 & 0 & 0\\
                                -1 & -2 & -1
                            \end{bmatrix}
                            *\boldsymbol{h}_{i,j}
            \end{split}\\
            S_g(x,y) &= C_g\sqrt{G_x(x,y)^2 + G_y(x,y)^2} \label{eq:sobelsg}
        \end{align}

        \begin{center}
            Where \(C_g\) is the weighting constant associated with the gradient score, \(S_g\).

            The \(*\) operator indicates the dot product.
        \end{center}

    \subsection{Terrain Proximity Score}
    The proximity score aims to bias the selected anchor point away from points near steep inclines in the terrain. This score is defined as the average of the height
    difference of the current point weighted by the gaussian kernel \(\boldsymbol{K}\), of size \(n\) by \(n\). The distance around inclines that is rejected depends on the the
    chosen size of the kernel \(\boldsymbol{K}\), the standard deviation of \(\boldsymbol{K}\) and the height differences. This score is described in Equation
    \ref{eq:terrain_prox}.

    \begin{equation} \label{eq:terrain_prox}
        S_p(x,y) = C_p\left|\frac{\sum\left[\boldsymbol{K}*(\boldsymbol{h}_{i,j}-\boldsymbol{h}_{x,y})\right]}{n^2}\right|
    \end{equation}

    \begin{center}
        \(x-\frac{1}{2}(n-1)<i<x+\frac{1}{2}(n-1)\) and \(y-\frac{1}{2}(n-1)<j<y+\frac{1}{2}(n-1)\)

        \vspace{0.2cm}
        Where \(x\) and \(y\) are the indices of the cell whose score is currently being evaluated, \(\boldsymbol{h}\) is the
        heightmap and \(C_p\) is the score's weighting constant.

        The \(*\) operator indicates the dot product.
    \end{center}
    
    \noindent
    A diagrammatic, sliced, representation of the proximity score can be seen in Figure \ref{fig:prox_score_diagram}. The
    slice is taken along the x axis of the heightmap.

    \subsection{Constraints}
    It is important to constrain the possible anchor points to confer to the stability triangle,
    meaning that the centre of mass of the robot must be inside the triangle formed by the three
    anchor points. Additionally, it is important that the points selected are not too far away,
    both in the horizontal and vertical direction for the robot to reach.


\section{Method of adjustment}
    Once the heightmap has been processed into the score map, which is done by adding the
    slope score and terrain proximity score, the point with the best score must be found for every
    initial anchor point.
    The resolution of the heightmap is not very high and the adjusted anchor point is not
    allowed to deviate too far from the initial anchor point, additionally due to the parallel nature
    of the heightmap generation it is possible to score the entire heightmap with minimal cost.
    Thus, it was decided to not employ an optimisation algorithm, such as gradient decent or
    Bayesian, but rather to use a radial search algorithm. This algorithm progressively expands
    its searching radius over the square score grid until a valid score is found, at which point it
    terminates, thus ensuring that the closest valid point to the initial anchor point is selected.
    See Figure \ref{fig:score_search} for a diagram representing the search pattern for a 5 grid squares search area.
    Note that this search pattern will become inaccurate for large search areas, as the pattern steps
    in a square manner. This however is not much of a concern for smaller search areas.

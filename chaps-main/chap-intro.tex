\chapter{Introduction}

\section{Background}

Starting from the big picture, gradually narrow focus down to this project and where this report fits in.
Why this specific project/report is worthwhile.


\section{Research Goal}
Broken down into sub objectives

The objectives of the project (in some cases the objectives of the report). If necessary describe limitations to the scope.


\section{Methodology}
When deciding how to determine optimal end effector placement various sensing methods were considered, such as using a \ac{rgbd} camera to view the environment,
placing force sensors on the robots feet or measuring servo torque to determine when the feet were in contact with a surface. A previous paper by ...... used a \ac{rgbd} camera
by storing past snapshots to adjust the feet to the optimal height, it was decided that the primary sensing method for this thesis would also be a \ac{rgbd} camera
but instead of storing snapshots, a height map would be generated of the local environment. This would allow for more advanced methods of placement selection
and preliminary collision checking for leg movements.

The first step in realising this system was to construct a accurate simulation of the hexapod. The primary simulation packages that were considered are Gazebo, PyBullet and \ac{mujoco}.
Gazebo was a appealing choice due to the easy integration with ROS, however it was decided to use \ac{mujoco} since it was found to have a far superior contact physics simulation.

Once the hexapod was adequately modelled in \ac{mujoco} a tripod gait state machine, \ac{ik} system and control interface was implement, at this stage the hexapod was capable of walking
on flat terrain.

Next the the system to generate the height map was implemented, this entailed sampling the \ac{rgbd} camera and comparing cells in the height map against the depth buffer.
Once the height map was implemented it was possible to build the system responsible for end effector placement, this is covered in detail in \autoref{chap:effector-placement},
after which collision checking for the generated end effector motion was implement, ensuring that the hexapod does not get stuck on pieces of terrain.

With this the system was realise in simulation, next the system was implemented and tested on the physical robot, discussed in detail in \autoref{chap:hardware}

\section{Scope and Limitations}

\section{Thesis Outline}

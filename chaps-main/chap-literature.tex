\chapter{Literature review}
This chapter will discuss previous work done regarding various elements of the project, this includes the overarching method of placing end effectors on
uneven terrain, simulation environments used, localisation in 3D space, and so forth.

\section{End Effector Placement Method}

Among other research focused on hexapods, many focus on topics such as obstacle avoidance, climbing surfaces, confined surfaces and cargo transportation.
When focusing of terrain adaptation most often the use of sensors such as \ac{lidar}, torque, or touch are employed. Where usually the height of end effectors
are adjusted to the height of the terrain \cite{coelho2021trends}.

Some papers, such as \cite{homberger2017terrain} utilise stereoscopic vision, in addition to end effector height adjustment, also focus on surface material classifications based on which the virtual
stiffness of the impedance controller is adjusted.

The focus of this paper will be on end effector height and planar position adaptation through real time walkability classification of the environment. 
While only utilising an \ac{rgbd} camera as sensor

\section{Localisation and Mapping}

This project requires a system that will localise the robot within its environment, as the primary sensor used is an \ac{rgbd} camera various visual \ac{slam} systems 
were considered. ORB-SLAM 3, a optimisation-based, sparse map \ac{slam} system was chosen to be used. ORB-SLAM 3 maintains a sparse map, an atlas, of both active and
dormant features. This atlas is used to localise in the sparse map \citep{macario2022comprehensive}.

The implementation of a dense map to be used for end effector placement is discussed in \autoref{chap:mapping}.

\section{Simulation Environment}

The most popular physics simulators for robotics in recent times are Gazebo, \ac{mujoco} and CoppeliaSim (previously V-REP) \citep{Collins-2021}.
Gazebo and CoppeliaSim both have easy to use \ac{gui} interfaces and easy integration with \ac{ros}. \ac{mujoco} on the other hand does not have
a full \ac{gui} interface, only a simulation viewer, and does not have native \ac{ros} integration. Having said this \ac{mujoco} was found to be
the most accurate and fastest simulator when considering the use case of robotics \citep{Erez-2015}.

Considering that the only relevant downside to \ac{mujoco} is the lack of native \ac{ros} integration and the lack of a comprehensive \ac{gui},
which seeing as \ac{mujoco} has good python bindings, could be seen as a advantage, \ac{mujoco} was chosen as the simulator.

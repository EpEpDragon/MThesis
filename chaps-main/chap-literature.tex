\chapter{Literature review}
This chapter provides an overview of past research done regarding the control of hexapod movement and sensing methods. First a brief history of hexapods is presented
after which various terrain sensing and adaptation methods are presented.

\section{Hexapod history}
Hexapoda, Greek for "six legs" refers the group of arthropods possessing three pairs of legs. As an example see a flesh-fly in \autoref{fig:flesh-fly}.

\begin{figure}[h]
    \centering
    \begin{minipage}{.5\textwidth}
        \centering
        \includegraphics[height=4cm]{flesh-fly.jpg}
        \caption{A Flesh-fly}
        \label{fig:flesh-fly}
    \end{minipage}%
    \begin{minipage}{.5\textwidth}
        \centering
        \includegraphics[height=4cm]{circ-hexapod.jpg}
        \caption{A circular hexapod}
        \label{fig:circ-hexapod}  
    \end{minipage}
\end{figure}

In the context of robotics "Hexapod" is used to refer to any robot with six legs, the most common configuration of Hexapods are either a rectangular
layout with three legs on either side mimicking biological Hexapoda, or a circular design with radially symmetrical leg spacing,
as seen in \autoref{fig:circ-hexapod}

The hexapod possess the minimum number of legs to allow a naturally stable platform since while taking a step there can be upwards of three anchor points around the center of mass at all times.
This makes the hexapod hexapods an ideal platform to navigate complex terrain while maintain stability, without requiring advanced balancing control systems.

For a hexapod to walk it must lift some of its legs while bracing with others, the number of swinging to bracing legs, and how each is moved, is referred to as the walking "gait".
The chosen gait influences the speed and stability of the hexapod, the tripod gait is considered to be the most well rounded, having good speed and stability.
In the tripod gait three legs are bracing while the remaining three swing. A example of a more stable gait would be the One by One gait, where only one leg is moved at a time.

It is also possible to create a system where there is no predetermined gait, but rather the system determines the optimal legs to brace and swing depending on the current walking environment.

\section{Control}
Walking over rough terrain requires a control system to correctly actuate the hexapods legs. Various types of control schemes exist, the primary schemes are traditional controllers,
bio-inspired controllers and \ac{rl}. These three schemes are discussed below. Control trends can be seen in \autoref{fig:control-trends}
\begin{figure}[h]
    \centering
    \includegraphics[width=\textwidth]{contoroller-trends.png}
    \caption{Trends of hexapod control schemes \citep{coelho2021trends}}
    \label{fig:control-trends}
\end{figure}

    \subsection{Traditional}
    Traditional controllers rely on an exact mathematical model of the robot and \ac{ik} to calculate angular commands for all leg joints. This method of control is purely kinematic does not 
    take into account external forces applied to the robot, thus it does not inherently adjust to the environment.

    Instead of a purley kinematic model, a dynamic model can also be used. Using a dynamic model the forces acting on the robots legs are taken into account, usually acquired through torque measurements 
    from servos. By taking applied torque into account dynamic model controllers will intrinsically detect a deviation when an external force is applied to the robot or its legs and compensate appropriately.

    It should be noted that it is possible for a kinematic model controller to also adjust to external disturbances, but this is not intrinsic to the control model and requires additional control logic.
    
    \subsection{Bio Inspired}
    Bio inspired controllers attempt to mimic the neural structure of animals to achieve the same locomotion methods that they use. This is implemented through the use of a \ac{ann}
    If implemented successfully a bio inspired controller can be highly adaptable to the surrounding environment and is even able to adapt to damaged or missing legs.

    \subsection{\acl{rl}}
    \ac{rl} controllers are created through using trial and error to construct a neural net that minimises a cost function for a specific goal. This theoretically allows \ac{rl} controllers to adapt to any
    circumstances given enough time, allowing a very hight level of autonomy, as no prior direction is required. \ac{rl} controllers are though notoriously difficult to train properly, especially when the 
    amount of sensors and control outputs grow large, increasing the feature space. And event he most well trained \ac{rl} agent still has the possibility to exhibit inexplicable behaviour.
    
\section{Sensors}
    No matter the control scheme used, to know where to place its feet the robot requires sensor(s) to sense its environment in some way, this could be achieved through simple sensors such as servo torque or touch,
    as used in \cite{t}. Or more advanced methods such as vision or \ac{lidar} could be used, as shown in \cite{t}.

    Depending on the terrain navigation system it might be required to localise the robot in 3D space, for this it is possible to use external sensors such as a type of beacon (RF, Reflective, Ultrasonic), \ac{gps}
    or, through the use of a \ac{slam} system, internal sensors, such as vision could be used.

    The 

    \subsection{End Effector Placement Method}

    Among other research focused on hexapods, many focus on topics such as obstacle avoidance, climbing surfaces, confined surfaces and cargo transportation.
    When focusing of terrain adaptation most often the use of sensors such as \ac{lidar}, torque, or touch are employed. Where usually the height of end effectors
    are adjusted to the height of the terrain \cite{coelho2021trends}.

    Some papers, such as \cite{homberger2017terrain} utilise stereoscopic vision, in addition to end effector height adjustment, also focus on surface material classifications based on which the virtual
    stiffness of the impedance controller is adjusted.

    The focus of this paper will be on end effector height and planar position adaptation through real time walkability classification of the environment. 
    While only utilising an \ac{rgbd} camera as sensor

    \subsection{Localisation and Mapping}

    This project requires a system that will localise the robot within its environment, as the primary sensor used is an \ac{rgbd} camera various visual \ac{slam} systems 
    were considered. ORB-SLAM 3, a optimisation-based, sparse map \ac{slam} system was chosen to be used. ORB-SLAM 3 maintains a sparse map, an atlas, of both active and
    dormant features. This atlas is used to localise in the sparse map \citep{macario2022comprehensive}.

    The implementation of a dense map to be used for end effector placement is discussed in \autoref{chap:mapping}.

\section{Simulation Environment}

The most popular physics simulators for robotics in recent times are Gazebo, \ac{mujoco} and CoppeliaSim (previously V-REP) \citep{Collins-2021}.
Gazebo and CoppeliaSim both have easy to use \ac{gui} interfaces and easy integration with \ac{ros}. \ac{mujoco} on the other hand does not have
a full \ac{gui} interface, only a simulation viewer, and does not have native \ac{ros} integration. Having said this \ac{mujoco} was found to be
the most accurate and fastest simulator when considering the use case of robotics \citep{Erez-2015}.

Considering that the only relevant downside to \ac{mujoco} is the lack of native \ac{ros} integration and the lack of a comprehensive \ac{gui},
which seeing as \ac{mujoco} has good python bindings, could be seen as a advantage, \ac{mujoco} was chosen as the simulator.

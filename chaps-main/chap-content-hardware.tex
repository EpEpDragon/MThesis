\chapter{Hardware Implementation} \label{chap:hardware}
This chapter describes the process of implementing the system built in previous chapters on the physical robot. \ac{ros} is used 
for the hardware implementation, first the \ac{ros} data types are defined, after which the control and analysis system running on the base station
is described. Next the systems running on board the robot are described.

\section{ROS Nodes}
There are various ros nodes spread out across the base station, the Jetson Nano and the Teensy \ac{mcu}, figure \ref{fig:nodes} provides and overview
of the nodes and how they communicate with each other. Section \ref{sec:base_ros} and \ref{sec:on_board_ros} go into detail on the publishers and subscribers
on board the robot and on the base station.

\section{ROS Data Types}
Various custom \ac{ros} data types are defined to assist with communication, these data types are described in table \ref{tab:data_types}.
\begin{table}[h]
    \centering
    \begin{tabularx}{\textwidth}{| l | p{\widthof{float32\([2]\) walk\_dir}} | X |}
        \hline
        \textbf{Name} & \textbf{Type Definition} & \textbf{Description} \\ \hline
        % walk\_dir & Description & Data Type & Frequency \\ \hline
        Vector3 & float\([3]\) data. & A vector in 3D space.  \\
        \hline
        EffectorTargets & Vector3\([6]\) targets \newline 
                          bool\([6]\) swinging. & Data describing the targets of the robot's feet and which feet are swinging. \\
        \hline
        HexapodCommands & float32\([2]\) walk\_dir \newline
                          float32 speed \newline
                          float32 height & Data packet containing various command parameters for the robot. \\
        \hline
    \end{tabularx}
    \caption{\ac{ros} data type descriptions}
    \label{tab:data_types}
\end{table}

\newpage
\section{ROS Base Station} \label{sec:base_ros}
    The base station wirelessly communicates with the robot to send commands and receive data.
    For a description of the base station publishers see table \ref{tab:base_pubs} and for its subscribers see \ref{tab:base_subs}. Table \ref{tab:data_types} describes
    data types used.
    \begin{table}[h]
        \centering
        \begin{tabularx}{\textwidth}{| l | l | X | l |}
            \hline
            \multicolumn{4}{|c|}{\textbf{Base Station Publishers}} \\ \hline
            \textbf{Name} & \textbf{Data Type} & \textbf{Description} & \textbf{Frequency} \\ \hline
            % walk\_dir & Description & Data Type & Frequency \\ \hline
            command\_data & HexapodCommands & Various robot command parameters. & On change \\ \hline
            mode & Int32 & Specifies the oporationg mode of the robot. & On change. \\ \hline
        \end{tabularx}
        \caption{Base station publishers}
        \label{tab:base_pubs}
    \end{table}
    
    \noindent
    The robot currently only has two operating modes, torque cutoff mode and walking mode. The torque cutoff mode is the initial mode the robot is in,
    while in this mode the leg servos disable all torque control, thus entering a relaxed state. In walking mode the robot walks in the commanded direction
    whilst optimising its foot positions according to the terrain. If the robot encounters a piece of terrain for which no optimisation can be found the
    human controller will have to adjust the walking direction from the base station.

    \begin{table}[h]
        \centering
        \begin{tabularx}{\textwidth}{| l | l | X |}
            \hline
            \multicolumn{3}{|c|}{\textbf{Base Station Subscribers}} \\ \hline
            \textbf{Name} & \textbf{Data Type} & \textbf{Description} \\ \hline
            % walk\_dir & Description & Data Type & Frequency \\ \hline
            rgb\_data & Image & The processed color image from the robot. \\ \hline
            d\_data & Image & The processed color depth from the robot. \\ \hline
            hmap\_data & Image & The heightmap generated on the robot. \\ \hline
            LOGDATA & String & General logs from the robot. \\ \hline
        \end{tabularx}
        \caption{Base station subscribers}
        \label{tab:base_subs}
    \end{table}
    
    \noindent
    The only subscribers present on the base station are the processed camera images, heightmap and logs. These are all used to provide a interface from where
    the oporator can control the robot.

    \section{ROS On Board} \label{sec:on_board_ros}
    The hexapod has two computational units on board, first the Jetson Nano, which handles all high level operations, including heightmap generation and scoring,
    foot optimisation, maintain a walking gait and localisation using ORB-SLAM3. Secondly a Teensy2.0 \ac{mcu} handles low level operations, including interpolating feet movement paths
    and servo control. Table \ref{tab:jetson_pubs} to \ref{tab:teensy_subs} describe the \ac{ros} publishers and subscribers present on these two computational units.
    \begin{table}[h]
        \centering
        \begin{tabularx}{\textwidth}{| l | l | X | l |}
            \hline
            \multicolumn{4}{|c|}{\textbf{Jetson Publishers}} \\ \hline
            \textbf{Name} & \textbf{Data Type} & \textbf{Description} & \textbf{Frequency} \\ \hline
            % walk\_dir & Description & Data Type & Frequency \\ \hline
            effector\_targets & EffectorTargets & Data indicating which feet to move where, and what type of interpolation to use. & On change\\ \hline
            rgb\_data & Image & The processed color image from the \ac{rgbd} camera. & 15Hz. \\ \hline
            d\_data & Image & The processed depth image from the \ac{rgbd} camera. & 15Hz. \\ \hline
            hmap\_data & Image & The heightmap generated on the robot. & 15Hz. \\ \hline
            position & Vector3 & The localised position of the robot & 15Hz \\ \hline
            rotation & Quat & The localised rotation of the robot & 15Hz \\ \hline
        \end{tabularx}
        \caption{jetson publishers}
        \label{tab:jetson_pubs}
    \end{table}

    \noindent
    The camera data, heightmap data, position and rotation are published for display at the base station.
    While the effector targets are published for use on the Teensy to move the robot's feet to the optimised positions.
    \begin{table}[h]
        \centering
        \begin{tabularx}{\textwidth}{| l | l | X |}
            \hline
            \multicolumn{3}{|c|}{\textbf{Jetson Subscribers}} \\ \hline
            \textbf{Name}  & \textbf{Data Type} & \textbf{Description} \\ \hline
            % walk\_dir & Description & Data Type & Frequency \\ \hline
            command\_data & HexapodCommands & Commands from the base station. \\ \hline
            color/image\_raw & Image & Raw color image from the \ac{rgbd} camera. \\ \hline
            aligned\_depth\_to\_color/image\_raw & Image & Raw depth image from the \ac{rgbd} camera. \\ \hline
        \end{tabularx}
        \caption{Jetson subscribers}
        \label{tab:jetson_subs}
\end{table}
As can be seen from table \ref{tab:jetson_subs} the only subscribers required on the Jetson is the raw camera feed, 
for constructing the heightmap, and the commands from the base station.

Table \ref{tab:teensy_pubs} show that the Teensy publishes the current feet positions these are the positions calculated through \ac{fk}. Additionally
log data is also published for use on the base station.
\begin{table}[h]
    \begin{tabularx}{\textwidth}{| l | l | X | l |}
        \hline
        \multicolumn{4}{|c|}{\textbf{Teensy Publishers}} \\ \hline
        \textbf{Name} & \textbf{Data Type} & \textbf{Description} & \textbf{Frequency} \\ \hline
        LOGDATA & String & General logs. & 10Hz \\ \hline
        effector\_current\_position & Eigen::Vector3d & Current feet positions. & 10Hz \\ \hline
    \end{tabularx}
    \caption{Teensy publishers}
    \label{tab:teensy_pubs}
\end{table}
\begin{table}[h]
    \centering
    \begin{tabularx}{\textwidth}{| l | l | X |}
        \hline
        \multicolumn{3}{|c|}{\textbf{Jetson Subscribers}} \\ \hline
        \textbf{Name} & \textbf{Data Type} & \textbf{Description} \\ \hline
        % walk\_dir & Description & Data Type & Frequency \\ \hline
        command\_data & HexapodCommands & The color image from the \ac{rgbd} camera. \\ \hline
        effector\_targets & EffectorTargets & Data indicating which feet to move where, and what type of interpolation to use.\\ \hline
        mode & Int32 & Receives mode data from the base station \\ \hline
    \end{tabularx}
    \caption{Teensy subscribers}
    \label{tab:teensy_subs}
\end{table}

\noindent
Lastly, from table \ref{tab:teensy_subs} it can be seen that the Teensy subscribes to the command data, effector targets and mode.
The walking speed component from the command data is used to set the rotational rate of the servos, as discussed in section \ref{sec:ang_rate}.
The effector targets are used to interpolate a curve for the feet to move along, as described in section \ref{sec:arc_generation}. The mode is required
as some modes could integrate directly with the servo control, namely the torque cutoff mode.

\section{GPU Processing}

% \bigskip
% \hrule
% \smallbreak
% \hrule
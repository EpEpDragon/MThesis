\chapter{Reference Frame Transforms} \label{app:transforms}
    Reference frame transforms are defined by the shorthand \(\transframe{\bm{x}\inrefframe{A}}{A}{B}\), meaning vector \(\bm{x}\inrefframe{A}\) is transformed from reference frame \refframe{A} to \refframe{B}. Reference frames used are the camera frame, \refframe{\fcamera}, the map frame, \refframe{\fmap}, the body frame, \refframe{\fbody}, the leg reference frames, \refframe{\fleg}, and the world frame, \refframe{\fworld}. Transforms used are descried in this appendix.

    \section{Camera to World}
        Transforming a vector from camera to map space requires rotating the vector by the camera quaternion and adding the camera's global position to the vector. As shown in equation \ref{eq:camera_to_map_a}.
        \begin{align} \label{eq:camera_to_map_a}
            \begin{split}
                \bm{p}\inrefframe{\fworld} &= \transframe{\bm{p}\inrefframe{\fcamera}}{\fcamera}{\fworld}\\ 
                &= \bm{q}_{\fcamera\fworld} \cdot \bm{p}\inrefframe{\fcamera} \cdot \bm{q}_{\fcamera\fworld}{-1} + \bm{p}\inrefframe{\fworld}_{C0}
            \end{split}
        \end{align}
        where \(\boldsymbol{p}\inrefframe{\fworld}\) is the position in the world frame, \(\boldsymbol{q}_{\fcamera\fworld}\) is the quaternion rotation of the world frame relative to the camera frame, and \(\boldsymbol{p}\inrefframe{\fworld}_{C0}\) is the position of the camera frame's origin in the world frame.
    
    \newpage
    \section{World to Map}
        The world to map transform is similar to camera to map tansform, except that the rotation is not required, as the map and world frames are already aligned.
        This is shown in equation \ref{eq:world_to_map}.
        \begin{align}\label{eq:world_to_map}
        \begin{split}
            \bm{x}\inrefframe{\fmap} &= \transframe{\bm{x}\inrefframe{\fworld}}{\fworld}{\fmap} \\
            &= \bm{x}\inrefframe{\fworld}\Delta \mod{N}
        \end{split}
        \end{align}
        where \(\Delta\) is the scaling factor used to relate heightmap cells to cm, and the heightmap buffer is of size \(N\times N\).
    \section{Local to Map}
        Local to map space is very similar to equation \ref{eq:world_to_map}, with the exception of using the body quaternion instead of the camera quaterion. This can be seen in
        equation \ref{eq:local_to_map}.
        \begin{equation} \label{eq:local_to_map}
            \bm{x}\inrefframe{\fmap} = (\bm{q_\text{bod}} \cdot \bm{x}\inrefframe{\fbody} \cdot \bm{q_\text{bod}}^{-1}\Delta + \bm{p_\text{rob}}\inrefframe{\fmap}) \mod N
        \end{equation}
        where \(\bm{x}\inrefframe{\fbody}\) is the vector in local space, \(\bm{q}_\text{bod}\) is the robot body's quaternion that rotates \(\bm{x}\inrefframe{\fbody}\) into map space, with its \(z\) axis pointing upwards, \(\bm{p_\text{rob}}\inrefframe{\fmap}\) is the coordinate vector of the robot in map space, and \(\Delta\) is the scaling factor
        used to relate heightmap cells to cm.

    \section{Map to Local}
        Due to the circular nature of the map buffer, translating from map to local space is more challenging. First an intermediate position, \(\bm{x_\text{temp}}\), is defined as a helper,
        \begin{equation} \label{eq:map_to_local_help}
            \bm{x_\text{temp}} = \bm{p_\text{rob}}\inrefframe{\fmap}\frac{1}{\Delta}
        \end{equation}
        where \(\bm{x_\text{temp}}\) is the map coordinates with the inverse map scaling applied. Now \(\bm{x_\text{temp}}\) must be checked to see if either of its 
        coordinates exceed half the map extents, if so, then it must be adjusted such that both coordinates fall within \(\frac{1}{2}E\).
        This is expressed in equation \ref{eq:map_to_local_adj}
        \begin{equation} \label{eq:map_to_local_adj}
            \bm{x_\text{temp}} \Leftarrow 
            \begin{cases}
                \bm{x_\text{temp}} & \bm{x_\text{temp}} \leq \frac{1}{2}E \\
                \bm{x_\text{temp}} - E(\sgn \bm{x_\text{temp}}) & \bm{x_\text{temp}} > \frac{1}{2}E
            \end{cases}
        \end{equation}
        where \(\sgn\) is the signum function, which return either 1 or -1 depending on the sign of the operand. 
        \textbf{Note} that equation \ref{eq:map_to_local_adj} is applied to both coordinates in the vector \(\bm{x_\text{temp}}\) separately, in pursuit of readability this is not explicitly stated.

        Now that \(\bm{x_\text{temp}}\) has been corrected for the circular nature of the map buffer, only the rotating into the local space remains. As show in equation \ref{eq:map_to_local_rot}.
        \begin{align} \label{eq:map_to_local_rot}
        \begin{split}
            \bm{x}\inrefframe{\fbody} &= \transframe{\bm{x}\inrefframe{\fmap}}{\fmap}{\fbody} \\
            &= \bm{q_\text{bod}}^{-1} \cdot \bm{x_\text{temp}} \cdot \bm{q_\text{bod}}
        \end{split}
        \end{align}
        where \(\bm{q}_\text{bod}\) is the robot body's quaternion.
    
    \section{Body to Leg}
    The body to leg transformation is a very simple transformation, consisting of only a rotation. This is because
    the body and leg frames are aligned to the same z axis. As there are six distinct leg coordinate frames the leg frame is
    annotated as, \(\fleg\), with the subscript \(i\) indicating the frame's associated leg number. Equation \ref{eq:body_to_leg_a}
    shows this transformation.
    \begin{equation}\label{eq:body_to_leg_a}
        \begin{split}
            \bm{x}^\fleg &= \transframe{\bm{x}^\fbody}{\fbody}{\fleg} \\
            & = \bm{Q}_i^\fbody\cdot\bm{x}^\fbody\cdot\bm{Q}^{\fbody^{-1}}_{i} - \bm{R}_i
        \end{split}
        \end{equation}

    \noindent
    where \(\bm{Q}_i^\fbody\) is the rotation quaternion of leg \(i\).
    
    Note that this rotation need not be represented as a
    quaternion, as it is a simple single axis rotation, but as it is entirety possible to have a leg axis system with a multi axis
    rotation, and quaternions are used throughout this paper, it was decided to use quaternions.
% !TeX encoding = UTF-8
% !TeX program  = xelatex

\documentclass[12pt,a4paper,openany,
               afrikaans,UKenglish,
               masters-t,goldenblock 
              ]{stb-thesis} 
% \usepackage{parskip}
%==== Language setup =================================================
\usepackage{babel}
\usepackage[utf8]{inputenc}%...................... Unicode input file format

%==== Math setup =====================================================
\usepackage{amsmath}%............................. Advanced math (before fonts)
%\usepackage{amssymb}%............................ AMS Symbol fonts
\usepackage{scalerel}
\usepackage{stackengine}
\DeclareMathOperator{\sgn}{sgn}
\DeclareMathOperator{\avg}{avg}
% \newcommand{\transframe}[2]{\ifmmode \quad (\mathcal{#1} \rightarrow \mathcal{#2}) \else \((\mathcal{#1} \rightarrow \mathcal{#2})\)  \fi}
\newcommand{\transframe}[3]{\ifmmode T_{#2#3}(#1) \else\(T_{#2#3}(#1)\)\fi}
\newcommand{\inrefframe}[1]{^\mathcal{#1}}
\newcommand{\mdim}[3]{\underset{\scriptscriptstyle#2 \times #3}{#1}}
\newcommand{\seq}[3]{\big[\mkern-4.7mu\big[#1,\:\dots\:,#3\big]\mkern-4.7mu\big]}
\newcommand{\refframe}[1]{\ifmmode \mathcal{#1} \else\(\mathcal{#1}\)\fi}
\newcommand{\fworld}{\refframe{W}}
\newcommand{\fbody}{\refframe{B}}
\newcommand{\fleg}{\refframe{L_\mathnormal{i}}}
\newcommand{\fcamera}{\refframe{C}}
\newcommand{\fmap}{\refframe{M}}



%==== Font setup (default is Computer Modern) ========================
\usepackage{iftex}
\ifxetex
    \usepackage[math-style=TeX,
                bold-style=TeX,
               ]{unicode-math}
    \setmainfont{Cambria}%........................ Unicode fonts  (Win)                
    \setsansfont[Scale=MatchLowercase]{Calibri}
    \setmonofont[Scale=MatchLowercase]{Consolas}
    \setmathfont{Cambria Math}
    \defaultfontfeatures{Ligatures=TeX}
    \let\bm\symbfit
\else
    \usepackage[utf8]{inputenc}%.................. Unicode file format
    \usepackage{textcomp}%........................ Additional text symbols
    \usepackage[T1]{fontenc}%..................... Type 1 outline fonts
    \usepackage{bm}%.............................. Bold math fonts
\fi
\normalfont
\usepackage{amsfonts}

%==== Units and numbers ==============================================
\usepackage{siunitx}%............................. Unit, number and angle output
    \sisetup{detect-all = true, detect-family = true}
    \sisetup{%output-decimal-marker = {.} ,
             group-separator = {\,},
             number-unit-product = {\,},
             inter-unit-product = \mathord{\cdot},
             exponent-product = \mathord{\times},
             separate-uncertainty = true}
   
         
%==== Ref's, Bib's and Nomencl =======================================
\usepackage{stb-nomencl}%......................... List of symbols 
    \renewcommand*{\UnitLabel}[1]{~[\,$\unit{#1}$\,]}
\usepackage{stb-bib}%............................. Bibliography (natbib internally)
    \bibliographystyle{stb-bib-eng-a}
    \renewcommand\bibfont{\small}
    \renewcommand\bibsection{\chapter{\bibname}}
\usepackage[printonlyused]{acronym}
\usepackage{hyperref}
\hypersetup{
    colorlinks,
    citecolor=black,
    filecolor=black,
    linkcolor=black,
    urlcolor=black
}
    
%==== Tables + Graphics + Color =====================================
\usepackage{array}%............................... Extended table defs 
    \setlength{\extrarowheight}{2pt}
\usepackage{longtable}%........................... Tables can break over pages
\usepackage{graphicx}%............................ Included graphics
\usepackage[font=small]{caption}%................. Customize captions  
\usepackage[table]{xcolor}%....................... Color setup + colortbl 
% \usepackage{pdfpages}
\usepackage{layouts}
\usepackage{tabularx}
% \newcolumntype{f}{p}
\usepackage{subcaption}
\usepackage[export]{adjustbox}
\graphicspath{ {./figs} }
\usepackage{float}
    
%==== Extra defs for template ========================================
\makeatletter
%---- TOC entries and case
    \addto{\captionsafrikaans}{\renewcommand\bibname{Lys van verwysings}}
    \addto{\captionsafrikaans}{\renewcommand\contentsname{Inhoudsopgawe}}
    \addto{\captionsafrikaans}{\renewcommand\listfigurename{Lys van figure}}
    \addto{\captionsafrikaans}{\renewcommand\listtablename{Lys van tabelle}}

    \addto{\captionsUKenglish}{\renewcommand{\bibname}{List of references}}
    \addto{\captionsUKenglish}{\renewcommand\contentsname{Table of contents}}
    \addto{\captionsUKenglish}{\renewcommand\listfigurename{List of figures}}
    \addto{\captionsUKenglish}{\renewcommand\listtablename{List of tables}}

%==== User Defs ======================================================
%
% Please insert user defined commands here
% and NOT in the document itself!
%

\makeatother

%==== Title Page =====================================================
\title{\bfseries
       \AorE{%-- Afrikaans ------------------------------------------
                Voet Plaasing Beplanning  van 'n Sespotige Robot oor Onegalige asdasd
            }{%-- English -------------------------------------------
                Foot Placement Planning for a Hexapod Robot Moving Over Uneven Terrain}}

\author{A.P.\ Lotriet}{Andries Phillipus Lotriet}

\degree{\AorE{MIng (Meg)}{MEng (EE)}}
       {\AorE{Magister in Ingenieurswese (Elektronies)}
             {Master of Engineering (Electronic)}}

\address{\AorE{%-- Afrikaans ----------------------------------------
             }{%-- English ------------------------------------------
        Department of Electrical and Electronic Engineering,\\
        Stellenbosch University,\\
        Private Bag X1, Matieland 7602, South Africa.}}

\faculty{\AorE{Fakulteit Ingenieurswese}%
              {Faculty of Engineering}}

\supervisor{Prof.\ J.A.A.\ Engelbrecht}
% \cosupervisor{Prof.\ J.\ Smith}

\setdate{12}{2024}

%\SetSponsor{The financial assistance of the National Research Foundation (NRF)
%    towards this research is hereby acknowledged. Opinions expressed and
%    conclusions arrived at, are those of the author and are not necessarily to
%    be attributed to the NRF.}


%==== Main Document ==================================================
\setcounter{secnumdepth}{3}
\setcounter{tocdepth}{2}
\raggedbottom
\begin{document}

\frontmatter%---------------------------------------------------------                    
\TitlePage

\DeclarationDate{2024/09/15}
\DeclarationPage


% \begin{abstract}[english]%===================================================
\chapter*{Abstract}
In recent times great strides have been made in the field of autonomous robotics, 
especially with regards to autonomous navigation of wheeled and aerial drones.
Legged robotics however still face numerous problems before they can become practical
to use, the most egregious of these problems being balancing of the robot, and optimal foot placement.

This thesis focuses on providing a solution to the latter problem of foot placement. This is achieved by using a depth camera to, in real time, construct a localised map of the environment, and subsequently analysing  said map for optimal foot placement locations. The system is then tested using a hexapod robot, both in simulation and on a physical robot.
% \end{abstract}


\chapter*{Uittreksel}
% \begin{abstract}[afrikaans]%=================================================
    In onlangse tye is groot vordering gemaak in die gebied van outonome robotika, 
    veral met betrekking tot outonome navigasie van hommeltuie. Bebeende-robotika het egter steeds probleme om op te los voordat dit prakties gebruik kan word, die mees ernstige van hierdie probleme is balansering van die robot, en optimale voetplasing.

    Hierdie tesis fokus daarop om 'n oplossing vir die laasgenoemde probleem van voetplasing voor te stel. Dit word bereik deur 'n dieptekamera te gebruik  om 'n gelokaliseerde kaart van die omgewing te konstrueer, en daarna die kaart te ontleed vir optimale voetplasings areas. Die stelsel word dan getoets met behulp van 'n seskantige-robot, beide in simulasie en op 'n fisiese robot.
% \end{abstract}

\chapter*{Acknowledgments}
    I would like to thank my supervisor, Prof. J.A.A Engelbrecht for his invaluable support and guidance throughout the course of this project. Additionally, I would like to thank my family and friends for their support and taking a keen interest in my work.

% \include{chaps-front/chap-dedication}

% Use \chapter*{} before TOC
\tableofcontents
% Use \chapter{} after TOC

\listoffigures
\listoftables
\chapter{List of symbols}
% Use stb-nomenclature + siunitx

\begin{Nomencl}[1cm]
\NomGroup{Constants}%-----------------------------------------------
    \item[$L_0 = $] \qty{300}{mm}

\NomGroup{Variables}%-----------------------------------------------
    \item[$\mathit{Re}_\mathrm{\,D}$]
                       \UnitLine{Reynolds number (diameter)}{~}
    \item[$x$]         \UnitLine{Coordinate                }{m}
    \item[$\ddot{x}$]  \UnitLine{Acceleration              }{m/s^2}\\
    
    \item[$\theta$]    \UnitLine{Rotation angle            }{rad}
    \item[$\tau$]      \UnitLine{Moment                    }{\newton\meter}

\NomGroup{Vectors and Tensors}%-------------------------------------
    \item[$\overrightarrow{\bm{v}}$] Physical vector, see equation ...

\NomGroup{Subscripts}%----------------------------------------------
    \item[$\mathrm{a}$] Adiabatic
    \item[$a$]          Coordinate

\end{Nomencl}

% \begin{Nomencl}[1cm]
% \NomGroup{Abreviations}%-----------------------------------------------
    % \item[DEM] Discrete Element Method
    % \item[FEA] Finite Element Analysis
    \begin{acronym}[MMIIII]
    \NomGroup{Abreviations}%-----------------------------------------------
        \acro{ik}[IK]{Inverse Kinematics}
        \acro{sdf}[SDF]{Signed Distance Field}
        \acro{mujoco}[MuJoCo]{Multi-Joint dynamics with Contact}
        \acro{gui}[GUI]{Graphical User Interface}
        \acro{ros}[ROS]{Robot Operating System}
        \acro{lidar}[LiDAR]{Light Detection and Ranging}
        \acro{rgbd}[RGB-D]{Red Green Blue Depth}
        \acro{slam}[SLAM]{Simultaneous Localisation and Mapping}
        \acro{imu}[IMU]{Inertial Measuring Unit}
    \end{acronym}    
% \end{Nomencl}



\mainmatter%----------------------------------------------------------
\numberwithin{figure}{chapter}
\numberwithin{table}{chapter}

\chapter{Introduction}

\section{Background}

    There are many applications where vehicles are required to traverse uneven terrain, such as in mines, rescue operations, agriculture, construction, etc. In many of these
    use cases uneven terrain makes the use of wheeled, or even tracked, vehicles difficult or impractical.

    Compared to wheeled robots, legged robots could perform better in many of these environments, allowing navigation over terrain that would be impossible for wheeled or
    tracked vehicles to navigate. While legged robots possess extreme degrees of potential terrain traversability, advanced control and sensory systems are required to 
    realise this potential.


\section{Research Goal}
    The overarching goal of this project is to design and implement a sensory, and control system that will allow a hexapod robot to autonomously walk over uneven terrain.

    This goal of the project is broken up into the following sub objectives:

    \begin{enumerate}
        \item Mathematically model the robot, its actuators, and its sensors.
        \item Create a simulation model of the robot in a suitable simulation environment for development and testing.
        \item Implement and test a baseline motion control system that enables the robot to
        perform a normal walking motion over flat, featureless terrain.
        \item Develop a real-time vision-based dense mapping system that enables the robot to sense and create a three-dimensional 
        dense map of the surrounding terrain in real time using images obtained from its onboard RBG-D camera.
        \item Implement a vision-based \ac{slam} system that enables the robot to localise itself relative to the terrain
        while simultaneously creating the dense map of the terrain.
        \item Develop a foot placement planning method that analyses the dense terrain map and determines suitable positions
        for the hexapod to place its feet while walking over uneven terrain.
        \item Extend the baseline motion control system to enable the robot to perform a modified walking motion over uneven terrain.
        \item Implement and test the system in simulation and on the physical hexapod robot.
    \end{enumerate}


\section{Methodology}
    When deciding how to determine optimal foot placement, various sensing methods were considered, such as using a \ac{rgbd} camera to view the environment, placing force sensors on the robot's feet or measuring servo torques to determine when the feet are in contact with a surface. Since a previous paper by \cite{erasmus2023guidance} used a \ac{rgbd} camera by storing past image snapshots to adjust the feet to the correct height, it was decided that the primary sensing method for this thesis would also be a \ac{rgbd} camera. Instead of storing previous images, an onboard heightmap would be generated of the local environment. This would allow for more advanced methods of selecting foot target positions.

    The first step in realising this system was to construct a representative simulation of the hexapod. The primary simulation packages that were considered are Gazebo, PyBullet, and \ac{mujoco}. Gazebo was an appealing choice due to its easy integration with \acf{ros}. However it was decided to use \ac{mujoco} since it was found to have superior contact physics simulation \citep{Erez-2015}.

    Once the hexapod was adequately modelled in \ac{mujoco}, a baseline motion control system for the hexapod to walk on flat terrain. The baseline control system included a tripod gait state machine,
    a foot position and trajectory planner, and leg motion controllers based on the inverse kinematics. The control interface allowed a user to command the hexapod to walk at a commanded speed in a commanded direction.

    Next, the the system to generate the heightmap from the images taken by the RGB-D camera was implemented. This entailed using the depth information in the \ac{rgbd} images to update the occupancy information in the cells of the height map.Once the height map was implemented, the system to perform foot placement planning on uneven terrain was developed.     The foot placement planning system operates by using parallel image processing to generate a walkability score map from the heightmap, which takes into account a cell's steepness and a cell's proximity to steep terrain. The nominal foot positions generated by the baseline motion system are then shifted to a location on the score map with a acceptable walkability score. This shifted position is used as the new, optimised foot target.

    The integrated system was first implemented and tested on the simulation model of the hexapod robot in MuJuCo. The mapping component of the system was then implemented and practically tested on the physical hexapod robot.

\section{Scope and Limitations}

    % As the hardware used was developed by \cite{erasmus2023guidance} the project will focus only on developing the necessary software to control the robot hardware.

    % The primary systems developed in this paper are the mapping and terrain scoring systems, while lower level systems such as foot arc generation and kinematics are also
    % implemented. 
    
    The purpose of the system is to perform foot planning to allow the hexapod to move over uneven terrain. The system is not expected to provide obstacle avoidance or high-level path planning to avoid hazardous terrain.

    The user interface to the system is a velocity command to the hexapod which specifies the speed and direction in which the hexapod must walk. The hexapod is to walk in a straight line at the velocity commanded by the user, while adjusting its foot placements to compensate for the terrain. The scope of the project does not include a waypoint navigation system. However, if the hexapod is able to follow velocity commands while adjusting its feet for the terrain, the system could easily be extended to also perform waypoint navigation.

    The feet must be placed on suitable surfaces that can support the hexapod. Foot placements positions are considered to be suitable if the foot placement area is relatively flat and not too steeply inclined, and the foot will not be placed in close proximity to a steep edge, and the height of the terrain at the intended foot placement position is known. The last requirements is intended to prevent the hexapod from stepping on surfaces it has not seen yet, or from stepping into holes in the terrain.

    The hexapod is not expected to identify unstable terrain or hazardous terrain types, such as loose
    earth or pools of water. This could be the topic of future projects.

    The hexapod assumes that the uneven terrain is navigable, and trusts that the user will not steer
    it to navigate terrain that does not contain any suitable foot placement locations. If the system encounters
    terrain that the system cannot find valid foot placement solutions for, the robot will simply freeze in place and await an updated velocity command from the
    operator.

    The primary focus of the project is the development of the mapping, terrain scoring, and foot placement planning
    functions. A secondary focus is the development of the hexapod simulation model with representative contact physics.
    For the practical testing, the physical hexapod robot developed by \cite{erasmus2023guidance} was used, and the software
    that implements the  mapping, scoring, and foot placement planning was added. The \ac{slam} function was implemented
    by using the ORB-SLAM3 software library, which was developed by \cite{campos2021orb}. The ORB-SLAM3 library only performed
    the sparse mapping necessary for localisation.

    It is assumed that the terrain sensing is limited to the images obtained from the RGB-D camera. Other terrain
    sensing methods, such as contact sensors on the feet, or torque sensors on the servo motors, are not available.
    If a leg were to collide with terrain due to inaccuracies in the terrain map, the robot would not adjust the leg trajectory.

    It is assumed that the hexapod robot is not equipped with an \ac{imu}. The stability of the
    robot on the terrain therefore depends on the accuracy of the heightmap, and the hexapod's pose estimate depends
    entirely on the visual SLAM system.

\section{Thesis Outline}

    Chapter 2 provides a literature review on the methods of control, sensing and simulation used for hexapod robots.

    Chapter 3 provides an overview of the physical hexapod robot that was used for the project, and the simulation model that was created to
    support the development and testing of the system and the modelling thereof. The overview includes the robot's mechanical hardware,
    onboard computer and sensors, and the simulation environment that was used.

    Chapter 4 describes the mapping system that was developed, including the dense height map to represent
    the local terrain about the hexapod, and the sparse SLAM system to localise the hexapod within the terrain

    Chapter 5 describes the baseline motion control system that enables the hexapod to walk over flat, featureless terrain.

    Chapter 6 describes the foot placement planning system that was developed to allow the hexapod to walk over
    uneven terrain. The system includes a terrain scoring system to analyse the terrain and an optimal search algorithm to find suitable foot placement positions

    Chapter 7 describes the hardware and software implementation of the system on the physical hexapod robot.

    Chapter 8 describes the final simulation tests that were performed on the system and presents the test results.

    Chapter 9 provides the conclusion of the research and recommendations for future work.

\chapter{Literature review}
This chapter will discuss previous work done regarding various elements of the project, this includes the overarching method of placing end effectors on
uneven terrain, simulation environments used, localisation in 3D space, and so forth.

\section{End Effector Placement Method}

Among other research focused on hexapods, many focus on topics such as obstacle avoidance, climbing surfaces, confined surfaces and cargo transportation.
When focusing of terrain adaptation most often the use of sensors such as \ac{lidar}, torque, or touch are employed. Where usually the height of end effectors
are adjusted to the height of the terrain \cite{coelho2021trends}.

Some papers, such as \cite{homberger2017terrain} utilise stereoscopic vision, in addition to end effector height adjustment, also focus on surface material classifications based on which the virtual
stiffness of the impedance controller is adjusted.

The focus of this paper will be on end effector height and planar position adaptation through real time walkability classification of the environment. 
While only utilising an \ac{rgbd} camera as sensor

\section{Localisation and Mapping}

This project requires a system that will localise the robot within its environment, as the primary sensor used is an \ac{rgbd} camera various visual \ac{slam} systems 
were considered. ORB-SLAM 3, a optimisation-based, sparse map \ac{slam} system was chosen to be used. ORB-SLAM 3 maintains a sparse map, an atlas, of both active and
dormant features. This atlas is used to localise in the sparse map \citep{macario2022comprehensive}.

The implementation of a dense map to be used for end effector placement is discussed in \autoref{chap:mapping}.

\section{Simulation Environment}

The most popular physics simulators for robotics in recent times are Gazebo, \ac{mujoco} and CoppeliaSim (previously V-REP) \citep{Collins-2021}.
Gazebo and CoppeliaSim both have easy to use \ac{gui} interfaces and easy integration with \ac{ros}. \ac{mujoco} on the other hand does not have
a full \ac{gui} interface, only a simulation viewer, and does not have native \ac{ros} integration. Having said this \ac{mujoco} was found to be
the most accurate and fastest simulator when considering the use case of robotics \citep{Erez-2015}.

Considering that the only relevant downside to \ac{mujoco} is the lack of native \ac{ros} integration and the lack of a comprehensive \ac{gui},
which seeing as \ac{mujoco} has good python bindings, could be seen as a advantage, \ac{mujoco} was chosen as the simulator.

% \include{chaps-main/chap-content-A}
\chapter{System Overview}
This chapter provides a high level overview of the hexapod, starting with the hardware, then the software, and finally the simulation
environment.
\section{Hardware}
The physical hexapod is primarily the same hexapod described in \cite{erasmus2023guidance}, this robot is shown in figure \ref{fig:hexapod}. For computation, a JetsonNano and Teensy2.0 \ac{mcu} is used. Actuation of the six, three degrees of freedom, legs is handled by 24 Dynamixel RX-64 servos. Sensing is handled by a Realsense D435i \ac{rgbd} camera, which includes a \ac{imu} (However the \ac{imu} is not utilised in this project.) Two Marvelmind ultrasonic \ac{lps} beacons are also present on the robot, although these are not used either.
\begin{figure}[h]
    \centering
    \includegraphics[height=6.6cm]{hexapod.png}
    \caption{Physical Hexapod}
    \label{fig:hexapod}
\end{figure}

\noindent
The only alterations made from the version describe in \cite{erasmus2023guidance} is the thickening of the wires powering the JetsonNano,
to prevent a voltage drop, and the replacement of the 3D printed legs with laser-cut aluminium legs, to prevent leg flexure.

A power cable is shown running to the left and going off image. All tests were conducted using 14.8V
from a bench power supply plugged into the battery port. This was done for convenience and can easily be swapped for a battery.

\section{Software}
The most basic software flow for a robot walking over flat terrain is shown in figure \ref{fig:basic_sys}. This system does not sense
its environment in any way and simply moves its feet along predetermined paths.
\begin{figure}[h]
    \centering
    % \hspace{-1.38cm}
    \includegraphics{Diagrams-SysDiffBlockBefore.drawio.pdf}
    \caption{Basic motion system operation.}
    \label{fig:basic_sys}
\end{figure}

\noindent
This basic system will work well enough for walking over flat terrain, but will struggle once any
deviation in terrain height is present. Thus, the proposed, more advanced system, operates with the flow shown in figure \ref{fig:adv_sys}.

The advanced system uses similar components as the basic system, but with the foot end position planner modified to incorporate
checks against a score map and a height map to validate, and if necessary, adjust the foot 
end positions to place the feet at suitable positions in the  terrain. If no suitable adjusted foot positions and trajectories
can be found given the terrain, then the hexapod does not execute the motion and freezes in place. The human operator or
high-level guidance system must then change the overall path of the hexapod. However, this is outside the scope of the current
project.

\newpage
\begin{figure}[h]
    \centering
    % \hspace{-1.38cm}
    \includegraphics{Diagrams-SysDiffBlockAfter.drawio.pdf}
    \caption{Advanced motion system operation.}
    \label{fig:adv_sys}
\end{figure}

\noindent
The goal of maneuvering uneven terrain is achieved by a combination of 4 primary systems, namely a mapping, 
foot placement optimisation, motion control and a localisation system. A high-level overview of the system implementation can be seen
in figure \ref{fig:system_diagram}.
\begin{figure}[h]
    \centering
    % \hspace{-1.38cm}
    \includegraphics[width=0.93\textwidth]{HexapodSystemDiagram.drawio.pdf}
    \caption{Physical system diagram}
    \label{fig:system_diagram}
\end{figure}

\noindent
The mapping system utilises the \ac*{rgbd} camera to construct a dense heightmap of the immediate surroundings of the robot. As the robot moves around, old data is erased to make way for new data. The size and resolution of the heightmap is adjustable to the available memory and computational power. The heightmap system is further covered in chapter \ref{chap:mapping}.

The foot placement optimisation system takes the heightmap as input and produces another map of equal size to the heightmap. This new map is the score map and is found by assigning a score to each cell of the heightmap. The score is dependant on how stable a position the cell would be for the robot to place its feet. The score map can then be used to evaluate, and adjust if necessary, the initial foot placement proposed by the motion system. The foot placement optimisation system is further covered in chapter \ref{chap:optimisation}.

All the movement of the robot is handled by the motion system, it is comprised of a gait state machine, a foot end position planner, a foot trajectory generator, and inverse kinematics. The gait state machine selects the swinging and supporting legs during for each step to achieve a tripod gait. The foot position planner calculates the nominal foot end positions based on the stepping parameters, namely stride length and step height, but without taking the terrain into account. Simple linear motion is not acceptable for the swinging feet, thus the foot arc generator produces a movement vector based on the remaining distance to a foot's destination, which if followed, results in an arc-like motion to the destination. Finally to execute any movements, positions must be converted to servo angle commands, and the foot velocities must be converted to servo angular rates, using the inverse kinematics equations. The motion system is covered in more detail in chapter \ref{chap:motion}.

% \newpage
\section{Simulation}
As said in section \ref{sec:sim_research}, various simulation environments were considered, but finally \ac{mujoco} was chosen due to its excellent contact physics simulation. The simulation of the hexapod includes the 24 servos, simulated as high gain, high damped angle controlled motors. The \ac{rgbd} camera is also simulated as a direct OpenGL rendering of the simulation environment. As the camera is a OpenGL rendering, all standard buffers used for rendering is generated, this includes the depth buffer. A SLAM system does not run in the simulation, rather simulated estimation noise, based on \cite{macario2022comprehensive}, is added to the true position and orientation directly taken from the simulation.

\newpage
\noindent
The software running on the simulation is largely equivalent to that running on the physical system, with only slight modification to integrate with the simulation instead of the hardware.
Figure \ref{fig:mujoco} shows a screenshot of the simulation environment
\begin{figure}[h]
    \centering
    % \hspace{-1.38cm}
    \includegraphics[width=\linewidth]{mujoco.png}
    \caption{The MuJoCo simulation environment}
    \label{fig:mujoco}
\end{figure}

% \bigskip
% \bigskip
% \hrule
% \smallbreak
% \hrule

\chapter{Mapping} \label{chap:mapping}
For accurate foot placement and localisation purposes the robot makes use of two maps, a sparse map covering a large area, and a dense map covering a small
area around the robot. The primarily use of the sparse map is for localisation and extracting pose data, i.e. orientation, velocity and rate. While the dense
map is used to analyse the terrain and find an appropriate point to place the three supporting feet.
It is possible to also use the sparse map for autonomous navigation, however this use case in not covered in this paper.
This chapter covers the design of the mapping system.

The localisation, sparse mapping and pose estimation is handle by ORB-SLAM3 as described in \cite{campos2021orb}. Since ORB-SLAM3 is not a system designed by the author, its
design will not be covered in this chapter. Implementation and operation details will however be covered in chapters \ref{chap:hardware} and \ref{chap:results}.

\section{Projection}
In order to generate a heightmap from a \ac{rgbd} image, it is first required to project the \ac{rgbd} image into 3D space, this is necessary because a heightmap is essentially a 3D environment,
that can be represented as a image due to the assumption of purely convex geometry. 

The camera can be described by its intrinsic and extrinsic parameters. Extrinsic parameters characterise the
cameras position in 3D space, and intrinsic parameters characterise the relationship between the image plane 3D space, 
assuming the camera is at the world origin and an zero rotation. \cite{hartley2003multiple}.

\newpage
\noindent
Refer to figure \ref{fig:projection} as a visual aid regarding projection. Note that this figure is drawn from the perspective of projecting from the image plane into the world,
if the objective was to project from the world onto the image plane the projection center and image plane would swap places, causing the image to be inverted, thus, this figure assumes
that the image rotation has been corrected.
\begin{figure}[h]
    \centering
    \includegraphics{Diagrams-Projection.drawio.pdf}
    \caption{Camera Projection}
    \label{fig:projection}
\end{figure}

\noindent
Together the extrinsic and intrinsic matrices form the projection matrix,
as shown in equation \ref{eq:projection_matrix},
\begin{equation} \label{eq:projection_matrix}
    \boldsymbol{P} = \boldsymbol{K}
    \begin{bmatrix}
        \boldsymbol{R} & \boldsymbol{T}
    \end{bmatrix}
\end{equation}
where \(\boldsymbol{K}\) is the intrinsic matrix and \(\begin{bmatrix} \boldsymbol{R} & \boldsymbol{T} \end{bmatrix}\) the extrinsic matrix, these are described
in equation \ref{eq:intrinsic} and \ref{eq:extrinsic}.

The projection matrix can be used to project a point on the image plane into world space as shown in equation \ref{eq:full_projection}.

\begin{equation} \label{eq:full_projection}
    \begin{bmatrix}
        u \\
        v \\
        1
    \end{bmatrix}
    = \boldsymbol{P}
    \begin{bmatrix}
        x \\
        y \\
        z \\
        1
    \end{bmatrix}
\end{equation}
where \(u,v\) are the pixel coordinates on the image plane and \(x,y,z\) are the coordinates in world space.
\begin{equation} \label{eq:intrinsic}
    \boldsymbol{K} =
    \begin{bmatrix}
        \alpha_x & \gamma   & u_0 \\
        0        & \alpha_y & v_0 \\
        0        & 0        & 1
    \end{bmatrix}
\end{equation}
where the focal length is represented by,
\[\alpha_x = f \cdot m_y\]
\[\alpha_y = f \cdot m_x\]
with \(m_x\) and \(m_y\) being the inverse of the width and height of a image plane pixel, \(f\) the focal length and \(u_0\),\(v_0\) being the principal point, optimally the center of the image plane.
The skew coefficient, \(\gamma\), is often, and in this case, 0.

The extrinsic matrix is as shown below,
\begin{equation}\label{eq:extrinsic}
    \begin{bmatrix}
        \boldsymbol{R} & \boldsymbol{T}
    \end{bmatrix}
    =
    \begin{bmatrix}
        \boldsymbol{R}_{3\times3} & \boldsymbol{T}_{3\times1} \\
        \boldsymbol{0}_{1\times3} & 1
    \end{bmatrix}
\end{equation}
where \(\boldsymbol{R}\) characterises the camera's heading in world space and \(\boldsymbol{T}\) the world origin expressed in 
the camera coordinate frame.

For ease of preprocessing points are first projected into the camera coordinate frame, in other words, the extrinsic matrix is omitted from equation \ref{eq:full_projection}.
The resultant matrix equation is shown in equation \ref{eq:local_projection}.

\begin{equation} \label{eq:local_projection}
    \begin{bmatrix}
        u \\
        v \\
        1 \\
        1/z
    \end{bmatrix}
    = \frac{1}{z}
    \begin{bmatrix}
        f_x & 0 & c_x & 0 \\
        0 & f_y & c_y & 0 \\
        0 & 0 & 1 & 0 \\
        0 & 0 & 0 & 1
    \end{bmatrix}
    \begin{bmatrix}
        x\\
        y\\
        z\\
        1
    \end{bmatrix}
\end{equation}
From equation \ref{eq:local_projection} \(x,y,z\) are found to be show in equation \ref{eq:proj_z} to \ref{eq:proj_y}.
\begin{align}
    z &= D_{u,v} \label{eq:proj_z}\\[0.2cm]
    x &= \frac{z(u - u_0)}{\alpha_x}\label{eq:proj_x} \\
    y &= \frac{z(v - v_0)}{\alpha_y}\label{eq:proj_y}
\end{align}
where \(D_{u,v}\) is the depth image pixel value at pixel coordinate \(u,v\).


\section{Memory}


\chapter{Baseline Motion System} \label{chap:motion}
    This chapter covers the systems governing the baseline motion of the robot, meaning the motion over simple flat terrain. 
    First the kinematics of the robot are defined, next the walking gait state machine is
    described, and finally the equations to defining a foots path of motion is described. An overview of these systems can be seen in figure \ref{fig:motion_system}.
    \begin{figure}[h]
        \centering
        % \hspace{-1.38cm}
        \includegraphics{Diagrams-MotionSystem.drawio.pdf}
        \caption{Motion System Overview}
        \label{fig:motion_system}
    \end{figure}
    \section{Overview}
        % This chapter describes the systems governing the motion of the robot, such as leg motion planning and gait generation.
        The basic operation of the motion system is as follows, first the robot is commanded to walk in a certain direction, at a
        certain speed and body height. These commands are sent from the base station to Jetson Nano on the robot,
        the motion controller node on the Jetson Nano then sends these commands to the gait state machine node, at a fixed frequency.
        The gait state machine uses the received direction stride length to generate leg states (swinging or supporting) and the ideal
        final position of each foot. These states and positions are sent back to the motion controller node where the positions are adjusted
        based on the heightmap data to ensure stable footing. The leg states and adjusted feet positions are then sent to the servo 
        controller node, this node controls the servos to move the robots feet to their final positions, either in a arc or linearly,
        depending on their state (swinging or supporting). %A high level diagram of the motion system can be seen in \ref{fig:motion_system}.

    \section{Kinematics}
        When commanding a foot position, the servo controller requires a function to calculate servo angles. While the foot arc planner, see section 
        \ref{sec:arc_generation}, requires the current position of the feet to function. The \ac{ik} and \ac{fk} functions described in this section provide
        this functionality. 
        
        \subsection{Coordinate Frames}
            The coordinate frames relevant to the kinematics of the robot are the body coordinate frame, \fbody, and the leg frame, \fleg. All foot targets/positions are specified
            if the body coordinate frame, while the kinematic systems operate in the leg coordinate frame. Thus a conversion from the body to leg frames is required. 
            Figure \ref{fig:coords_top} show the world, body and multiple leg coordinate frames.
            \begin{figure}[h]
                \centering
                % \hspace{-1.38cm}
                \includegraphics[width=.7\textwidth]{Diagrams-BodyFrame.drawio.pdf}
                \caption{World, body and leg coordinate frames.}
                \label{fig:coords_top}
            \end{figure}

            \noindent
            The leg coordinate systems are simply rotated and shifted from the body coordinate system. This transformation,\transframe{\bm{x}^\fbody}{\fbody}{\fleg} is defined by equation \ref{eq:body_to_leg},
            further explained explained in appendix \ref{app:transforms}.
            \begin{equation}\label{eq:body_to_leg}
            \begin{split}
                \bm{x}^\fleg &= \transframe{\bm{x}^\fbody}{\fbody}{\fleg} \\
                & = \bm{Q}_i^\fbody\cdot\bm{x}^\fbody\cdot\bm{Q}^{\fbody^{-1}}_{i} - \bm{R}_i
            \end{split}
            \end{equation}

            \noindent
            where the \(\fleg\) is the coordinate frame of leg \(i\), \(\bm{Q}_i^\fbody\) is the rotation of said coordinate frame and \(\bm{R}^\fbody_i\) is the root position of
            said leg coordinate frame in the body coordinate frame. For more detail on this transformation, such as the values of \(\bm{Q}_i^\fbody\) and \(\bm{R}^\fbody_i\), please see
            appendix \ref{app:transforms}.

            Now that positions can be transformed into the leg coordinate frame the kinematic equations, as described in the following sections, can be applied. The kinematic equations are defined
            with reference to the variables in the leg frame as shown in figure \ref{fig:kinematics}.
            \begin{figure}[h]
                \centering
                % \hspace{-1.38cm}
                \includegraphics[clip, trim=0 0 1.1cm 0]{Diagrams-Kinematics.drawio.pdf}
                \caption{Leg coordinate frame with kinematic variables.}
                \label{fig:kinematics}
            \end{figure}

        \newpage
        \subsection{\acf{ik}}
            The \ac{ik} function calculates the leg servo angles, \(\bm{\Theta} = [\Theta_1, \Theta_2, \Theta_3]^T_{\displaystyle ,}\) required
            to move the foot to the given target position vector, \(\bm{t} = [x_t,y_t,z_t]^T_{\displaystyle .}\)
            \hbox{Equation \ref{eq:ik}} describes the \ac{ik} function.
            \begin{equation}\label{eq:ik}
                \bm{\Theta}(x_t,y_t,z_t) =
                                    % \begin{bmatrix}
                                    %     \Theta_1\\
                                    %     \Theta_2\\
                                    %     \Theta_3
                                    % \end{bmatrix}
                                    % =
                                    \begin{bmatrix}
                                        \arctan{\left(\dfrac{x_t}{y_t}\right)}\\[0.5cm]
                                        \dfrac{\pi}{4} - \alpha - \arctan{\left(\dfrac{y_t}{d-L_1}\right)}\\[0.5cm]
                                        \dfrac{\pi}{2} - \beta
                                    \end{bmatrix}
            \end{equation}
            where \(\alpha\), \(\beta\), \(c\) and \(d\) are calculate as shown in equations \ref{eq:alpha} to \ref{eq:dik}. For derivations of these variable
            please see \ref{app:kinematic}.
            \begin{align}
                \alpha &= \arcsin{\left(\frac{L_3\sin{\beta}}{c}\right)} \label{eq:alpha} \\[0.5cm]
                \beta &= \arccos{\left(\dfrac{L_1^2 + L_2^2 -c^2}{2L_1L_2}\right)}\\[0.5cm]
                c &= \sqrt{(d-L_1)^2+z_t^2}\\[0.5cm]
                d &= \sqrt{x_t^2 + y_t^2} \label{eq:dik}
            \end{align}
        \subsection{\acf{fk}}
            The \ac{fk} function calculates the position vector of a foot, \(\bm{p}_f = [x_f,y_f,z_f]^T\),
            given the current angles of the leg servos, \(\bm{\theta} = [\theta_1, \theta_2, \theta_3]^T\).
            \begin{align}
                \bm{p}_f(\theta_1,\theta_2,\theta_3) =
                                % \begin{bmatrix}
                                %     x_c\\
                                %     y_c\\
                                %     z_c
                                % \end{bmatrix}
                                % =
                                \begin{bmatrix}
                                    d\cos{\theta_1}\\
                                    d\sin{\theta_1}\\
                                    L_2\sin{\theta_2} + L_3\sin{\left(\theta_2 + \theta_3\right)}
                                \end{bmatrix}
            \end{align}
            where \(d\) is calculated as shown in in equation \ref{eq:dfk}.
            \begin{equation}\label{eq:dfk}
                d = L1 + L_2\sin{\theta_2} + L_3\sin{(\theta_2 + \theta_3)}
            \end{equation}
        
        \newpage
        \subsection{Angular Rate} \label{sec:ang_rate}
            To move a foot on a desired path it is important to not only know the absolute angle of the three leg servos, but also the angular rates of all three
            servos. If the servos are all moved at the same rate, the shape of the path that the foot follows will not be linear, but rather dependant on the
            current foot position. This is undesirable, thus equations \ref{eq:rate} define the derivative of the \ac{ik} equations (\ref{eq:ik}), i.e. the angular
            rate, given the target movement speeds of a foot, \(\dot{x}\), \(\dot{y}\) and \(\dot{z}\).

            \begin{equation}\label{eq:rate}
                \bm{\omega}(\dot{x}, \dot{y}, \dot{z}) =
                                    % \begin{bmatrix}
                                    %     \omega_1\\
                                    %     \omega_2\\
                                    %     \omega_3
                                    % \end{bmatrix}
                                    % =
                                    \begin{bmatrix}
                                        \dfrac{- x\dot{y} + y \dot{x}}{x^2 + y^2}\\[0.5cm]
                                        \dfrac{\left[(L_1 - d)\dot{z} + z\dot{d}\right]\alpha + \Big[(L_1 - d)^2 + z^2\Big]\arctan{\left(\dfrac{L_1-d}{z}\right)}\dot{\alpha}}{(L_1 - d)^2 + z^2}\\[0.8cm]
                                        -\dot{\beta} 
                                    \end{bmatrix}
            \end{equation}
            where \(\dot\alpha\), \(\dot\beta\), \(\dot{c}\) and \(\dot{d}\) as shown in equations \ref{eq:alphadot} to \ref{eq:cdot}.
            \begin{align}
                \dot{\alpha} &= \frac{ L_3\left[ c\cos(\beta)\dot{\beta} - \sin(\beta)\dot{c} \right] }{ c^2\sqrt{-\dfrac{L_3^2\sin^2(\beta)}{c^2}+1} } \label{eq:alphadot} \\[0.5cm] 
                \dot{\beta} &= \frac{ 2c\dot{c} }{ L_2L_3\sqrt{4 - \dfrac{(L_2^2+L_3^2-c^2)^2}{L_2^2L_3^2}} } \label{eq:betadot} \\[0.5cm]
                \dot{c} &= \frac{-(L_1 - d)\dot{d} + z\dot{z}}{\sqrt{(L_1 - d)^2 + z^2}} \label{eq:bdot} \\[0.5cm]
                \dot{d} &= \frac{x\dot{x} + y\dot{y}}{\sqrt{x^2 + y^2}} \label{eq:cdot}
            \end{align}
    
    \newpage
    \section{Walking Gait}
        To move the hexapod must support its body with some of its legs while the remaining legs swing towards their new targets,
        at which point the swinging legs become the new supporting. The sequence in which the legs support and swing is called the walking gait.
        Figure \ref{fig:gait_patterns} shows three different gait patterns that can be used with a hexapod, the wave ripple and tripod gait \citep{Darbha2017AnOS}.
        The dark cell represents a swinging leg and the light cell a supporting leg.
        \begin{figure}[h]
            \centering
            % \hspace{-1.38cm}
            \includegraphics{Diagrams-GaitDiagram.drawio.pdf}
            \caption{Three hexapod gait patterns.}
            \label{fig:gait_patterns}
        \end{figure}

        \noindent
        The wave gait moves one leg at a time while supporting with the remaining 5, the ripple gait moves two legs at a time, and the tripod moves
        three legs at a time.

        The speed of the hexapod is based on the parameters of the gait, specifically as described in equation \ref{eq:speed},
        \begin{equation} \label{eq:speed}
            v = \frac{S}{D\tau}
        \end{equation}
        where \(S\) is the stride length, \(tau\) is the gait period and \(D\) is the duty factor. \(D\) is defined as the time a leg is in the support phase relative to its swing phase.
        The wave, ripple and tripod gaits have a duty factor of \(\frac{5}{6}\), \(\frac{2}{3}\) and \(\frac{1}{2}\) respectively.
        
        From this it is clear that the wave gait is the slowest while the tripod gait is the fastest, and the ripple gait is in between.
        It should however be noted the gait's stability is inverse to their speed.

        The gait that will be used in this system is the tripod gait, which is the most common gait for hexapods as it supports with three legs, while maximising speed. Even though this is less stable than
        the wave and ripple gaits, it does maintain natural stability with three contact points, which is adequate for most circumstances.

        \subsection{Stride Reference Frame}
            When describing the stride of the robot it is important to note which reference frame is being used. Figure \ref{fig:stride_body} shows
            the stride of a tripod gait in the body reference frame, \(\fbody\).
            \begin{figure}[h]
                \centering
                % \hspace{-1.38cm}
                \includegraphics{Diagrams-StrideDiagramLocal.drawio.pdf}
                \caption{Hexapod stride relative to body coordinates.}
                \label{fig:stride_body}
            \end{figure}

            \noindent
            As can be seen, the swinging legs move in the direction of movement, following a arced path.
            While the supporting legs move in the opposite direction of movement following a linear path. 

            \newpage
            \noindent
            However, when looking at the same stride in the world reference frame, \(\fworld\), as can be seen in figure \ref{fig:stride_world},
            the supporting legs appear to stay stationary, while the swinging legs move double the distance relative to that in the body reference frame.
            \begin{figure}[h]
                \centering
                % \hspace{-1.38cm}
                \includegraphics{Diagrams-StrideDiagramGlobal.drawio.pdf}
                \caption{Hexapod stride relative to world coordinates.}
                \label{fig:stride_world}
            \end{figure}

            \noindent
            This is important to note because, as further discussed in section \ref{sec:choosing_nominal}, the nominal foot positions are chosen in the body reference frame, but must target a
            position in the world reference frame.
        
        \newpage
        \subsection{State Machine}
            The state machine used to realise the tripod gait used in the robot is quite simple, comprised of only two states, stepping and
            resting, as can be seen from figure \ref{fig:gaitSM}. Table \ref{tab:state_defs} defines the actions that should be taken during
            each state.

            The primary computation done by this state machine is calculating which legs are supporting and which are swinging, which occurs
            on entering the "Stepping" state. This function is described in section \ref{sec:supp_swing_calc}.

            \begin{figure}[h]
                \centering
                % \hspace{-1.38cm}
                \includegraphics{Diagrams-GaitSM.drawio.pdf}
                \caption{Gait State Machine}
                \label{fig:gaitSM}
            \end{figure}

            \begin{table}[h]
                \center
                \begin{tabularx}{\textwidth}{|l|X|}
                    \hline
                    \multicolumn{2}{|c|}{Rest State Definition} \\
                    \hline
                    Enter Condition & Has all feet reached their targets? \\
                    \hline
                    On Entering & Set all leg states as supporting. \\
                    \hline
                    While Active & Do nothing \\
                    \hline
                \end{tabularx}
                
                \bigskip
                \noindent
                \begin{tabularx}{\textwidth}{|l|X|}
                    \hline
                    \multicolumn{2}{|c|}{Stepping State Definition} \\
                    \hline
                    Enter Condition & Is there a mismatch between feet targets and current position? \\
                    \hline
                    On Entering & Calculate and set the leg states based on walking direction, see section \ref{sec:supp_swing_calc}. Choose and optimise nominal targets
                    for the current step, see chapter \ref{chap:optimisation}\\
                    \hline
                    While Active & Adjust feet targets based on direction, stride length and robot height, see chapter \ref{chap:optimisation}\\
                    \hline
                \end{tabularx}
                \caption{State Definitions}
                \label{tab:state_defs}
            \end{table}

        \newpage
        \subsection{Choosing The Supporting And Swinging Legs} \label{sec:supp_swing_calc}
            The robot body is divided up into sextants, centered around the nominal leg positions. When calculating
            the swinging legs it is first determined in which sextant the movement direction vector falls, this is called the active sextant.
            The leg related with the active sextant, and the two opposite, are then chosen as swinging, with the remaining three legs chosen as supporting.
            The states of the legs are encapsulated by the boolean array, \(\bm{S_i}\), defined by equation \ref{eq:is_swing},
            where a 1 indicates swinging and 0 supporting.
            \begin{equation}\label{eq:is_swing}
                \bm{S}_{\bm{i} - \xi}=[i \text{ is even}]
            \end{equation}
            where \(\xi \in i\) is the active sextant/leg number, and,
            \[i = \seq{0}{1}{5}\]

            Figure \ref{fig:sextants} illustrates and example with sextant 1 being active. Thus legs 1, 3 and 5 are swinging, while legs 0, 2 and 4 are supporting.
            \begin{figure}[h]
                \centering
                \hspace{1.1cm}
                \includegraphics[clip, trim=0 0.25cm 0 0.25cm]{Diagrams-Sextants.drawio.pdf}
                \caption{Leg sextants, with sextant 1 being active.} 
                \label{fig:sextants}
            \end{figure}
            
            \noindent
            This of course would not be sufficient to define a walking gait, as at the end of each step \(\bm{S_i}\) does not invert.
            Thus and additional step after equation \ref{eq:is_swing} is added. The current horizontal length of leg \(\xi\), defined as \(l_\xi\), is compared to its nominal
            horizontal length, \(L_\xi\). If \(l_\xi\ > L_\xi\), invert \(\bm{S_i}\). As shown in equation \ref{eq:negate_is_swing}.
            \begin{equation} \label{eq:negate_is_swing}
                \bm{S}_{\bm{i} - \xi} =
                                                    \begin{cases}
                                                        \bm{i} \setminus \bm{S}_{\bm{i} - \xi} & l_\xi > L_\xi \\
                                                        \bm{S}_{\bm{i} - \xi} & l_\xi \leq L_\xi
                                                    \end{cases}\\[0.1cm]
            \end{equation}

            \newpage
        \subsection{Choosing nominal foot positions}\label{sec:choosing_nominal} 
            Once the active and inactive legs have been selected it is required to choose nominal targets for all the feet. Sections \ref{sec:support} and \ref{sec:swing}
            describe the process for finding the support and swinging leg targets; It should be noted that target matrices, denoted by \(\bm{t}\), are common between these two sections, but are computed
            differently depending on whether the leg is swinging or supporting.
                
            \subsubsection{For Supporting Legs} \label{sec:support}
                The supporing leg nominal trargets are chosen with using equations \ref{eq:move_vec} and \ref{eq:opt_supp}, with reference
                to figure \ref{fig:supp_targ}.
                \begin{figure}[h]
                    \centering
                    % \hspace{1.1cm}
                    \includegraphics[clip, trim=3cm 0.25cm 3.4cm 0.25cm]{Diagrams-ChooseNominalSupp.drawio.pdf}
                    \caption{Supporting target choosing diagram} 
                    \label{fig:supp_targ}
                \end{figure}

                \noindent
                First the required move vectors, \(\bm{m} ^\fbody\), are calculated in equation \ref{eq:move_vec},
                \begin{equation}\label{eq:move_vec}
                    \mdim{\boldsymbol{m}^\fbody}{3}{1} =  (\bm{T}^\fbody - l_\text{str} \bm{\bm{w}_\text{dir}}^\fbody) - \bm{t_\text{nom}}^\fbody
                \end{equation}
                where \(\bm{T}^\fbody\) contains the constant rest positions the feet, \(l_\text{str}\) is the desired stride length, \(\bm{w}_\text{dir}^\fbody\)
                contains the desired walk directions, and \(\bm{t}_\text{nom} ^\fbody\) contains the nominal targets as calculated in
                equation \ref{eq:swing_nom}.

                Then the optimised targets, \(\bm{t_\text{opt}} ^\fbody\), are calculated as the addition of the previous targets and the move vector in equation \ref{eq:move_vec},
                \begin{equation} \label{eq:opt_supp}
                    \mdim{\boldsymbol{t}_\text{opt}^\fbody}{3}{6} = \bm{t}_\text{prv}^\fbody + \bm{m}^\fbody
                \end{equation}
                where \(\bm{t}_\text{prv}^\fbody\) contains the previous, optimised, targets. Note that equation \ref{eq:opt_supp} does not include the optimisation
                function \textcolor{red}{XXXX}, this is because the supporting feet will not move relative to the terrain, and thus do not need optimising.

            \newpage
            \subsubsection{For Swinging Legs} \label{sec:swing}
                The swinging leg nominal targets are chosen with reference to figure \ref{fig:swinging_targ}.
                \begin{figure}[h]
                    \centering
                    % \hspace{1.1cm}
                    \includegraphics[clip, trim=3.25cm 0.25cm 4.9cm 0.3cm]{Diagrams-ChooseNominalSwing.drawio.pdf}
                    \caption{Swinging target choosing diagram} 
                    \label{fig:swinging_targ}
                \end{figure}

                \noindent
                The nominal targets in map space, \(\bm{t_\text{nom}} ^{\fmap}\), are calculated as in equation \ref{eq:t_map},
                \begin{equation} \label{eq:t_map}
                    \mdim{\boldsymbol{t}_\text{nom}^{\fmap}}{3}{6} = \transframe{\bm{T}^\fbody + \left(l_\text{str}+|\bm{m}^\fbody|\right)\bm{w}_{dir}^\fbody}{\fbody}{\fmap}
                \end{equation}
                where \(\bm{T}^\fbody\) contains the constant rest positions the feet, \(l_\text{str}\) is the desired stride length, \(\bm{w}_\text{dir}^\fbody\)
                contains the desired walk directions. \(\avg\bm{m}^\fbody\) is the average move vector 

                For use in equation \ref{eq:move_vec}, the swinging legs nominal targets, \(\bm{t_\text{nom}} ^\fbody\), are calculated as 
                in equation \ref{eq:swing_nom},
                \begin{equation} \label{eq:swing_nom}
                    \mdim{\boldsymbol{t}_\text{nom}^\fbody}{3}{6} = \bm{T}^\fbody + l_{str}\bm{w}_\text{dir}^\fbody
                \end{equation}

                Note that \(\transframe{\bm{t}_\text{nom}^{\fmap}}{M}{L}\) is quite similar to \(\bm{t}_\text{nom}^\fbody\), the difference is that \(\bm{t}_\text{nom}^{\fmap}\) is set such that it
                will align with \(\bm{t}_\text{nom}^\fbody\)
                at the end of the step. Thus, the omission of the average supporting move vector, \(\avg \bm{m}^\fbody\), in calculating \(\bm{t}_\text{nom}^\fbody\)
                in equation \ref{eq:swing_nom}.
                Meaning at until the end of the step \(\bm{t}_\text{nom}\inrefframe{M}\) and \(\bm{t}_\text{nom}^\fbody\) will point 
                to different positions if world space. This is due to needing to account for body movement in world space, but not in local space.

        \newpage
    \section{Foot Motion} \label{sec:arc_generation}
        When taking a step the foot can not simply be moved to its destination in a straight line, as doing so will cause the foot to be dragged on the terrain,
        impeding the movement of the robot. Thus it is required to move the foot in an arc like motion to clear any obstacles that might be in its path.


        \subsection{Existing System}
            The existing system will, at the start of each step, compute an arc for each foot to follow, this arc is then sent to the servo controller
            to be executed. The arc is computed as a polynomial passing through three points, the initial point\(q_i\), the middle point \(q_m\), and the final point \(q_f\).
            Figure \ref{fig:old_arc_vars} shows the variables used to calculate this arc.
            \begin{figure}[h]
                \centering
                \includegraphics{Diagrams-ArcCalc.drawio.pdf}
                \caption{Variables for calculating foor arc.}
                \label{fig:old_arc_vars}
            \end{figure}

            \noindent
            Equation \ref{eq:old_arc} shows the calculations used to find the starting, middle and end points shown in figure \ref{fig:old_arc_vars}.

            \begin{align}\label{eq:old_arc}
                OldArcEquation\\
                OldArcEquation\\
                OldArcEquation\\
                OldArcEquation\\
                OldArcEquation\\
                OldArcEquation\\
                OldArcEquation\\
                OldArcEquation\\
                OldArcEquation
            \end{align}
            where etc\dots

            While efficient and effective in ideal conditions, this method of defining the arc has poor performance when considering external
            influences. If for example the robot has to adjust the final target of its feet mid step, this arc would have to be recomputed in its entirety,
            thus leading to possible performance concerns.

            In addition to this Text, the current system is designed with the assumption that the starting position of the foot is grounded, thus if the arc is recomputed
            mid step the arc will be undesirable, as it will rise with the desired step height for a second time. This is illustrated in Figure \ref{fig:old_arc}.

            \begin{figure}[h]
                \centering
                \hspace{-1.38cm}
                \includegraphics[clip, trim=0 0.25cm 0 0.25cm]{old_path.pdf}
                \caption{Existing arc recomputation problem}
                \label{fig:old_arc}
            \end{figure}

        \newpage
        \subsection{Improved System}
            The improved system solves this problem by utilising a flow function. During a step, this function will continuously calculate the
            direction that the foot must move to reach its destination. Thus this system is resilient to external disturbances and is capable of adjusting to
            varying destination and step height requirements. 
            
            The flow field is designed to first move the foot vertically upwards until horizontal coplanar with the destination, and then to follow a
            arc to the destination with a defined step height, this can be adjusted to make the arc start before or after coplanar. The step height can be adjusted at any point in time and the flow field will adjust accordingly.
            Figure \ref{fig:foot_arc} illustrates the field function and is described in section \ref{sec:flow_function}.
            \begin{figure}[h]
                \centering
                \hspace{-1.38cm}
                \includegraphics[clip, trim=0 0.25cm 0 0.25cm]{foot_path.pdf}
                \caption{End effector movement path}
                \label{fig:foot_arc}
            \end{figure}

            \subsubsection{Flow Function Description} \label{sec:flow_function}
                The flow function, \(\rho(x,y)\), uses the gradient function of a parabola passing through the point \([0,0]\) and \([x,y]\) as a basis, where point \([x,y]\)
                is the current point that is being evaluated and \(x\) is the horizontal distance between the destination and the current point and \(y\) the
                vertical distance. The final function is described by equations \ref{eq:rho} to \ref{eq:sigmoid}, for the process of designing the flow function
                please see appendix \ref{app:flow_function}.
                \begin{equation} \label{eq:rho}
                    \begin{aligned}
                        \rho(x,y) &= \frac{\delta}{\delta x\delta y}&&f_a(x,y)x^2 + f_b(x,y)x + C\\
                        &= &&2f_a'(x,y)x + f_b'(x,y)    
                    \end{aligned}
                \end{equation}
                where, %\(f_a'(x,y)\) and \(f_b'(x,y)\) are defined as follows:
                \begin{align} \label{eq:fa}
                    f_a'(x,y) &= -\left|\frac{v_h}{x}\right| - \left|S(y)\right|\\
                    f_b'(x,y) &= \frac{y}{x} - f_a(x,y)
                \end{align}
                with \(v_h\) being the variable describing the step height and \(S(y)\) being a sigmoid like function 
                responsible for the initial vertical rise. \(S(y)\) is defined in Equation \ref{eq:sigmoid}.
                \begin{equation} \label{eq:sigmoid}
                    S(y) = \frac{0.515(y-q)}{1+\left|y-q\right|-0.505}
                \end{equation}
                where \(q\) is the variable that determines at which vertical displacement the leg path transitions from primarily an vertical motion to
                an arc motion. Figure \ref{fig:sigmoid_like} illustrates the sigmoid like function for different values of \(q\).
                \begin{figure}[h]
                    \centering
                    \hspace{-1.38cm}
                    \includegraphics[clip, trim=0 0.25cm 0 0.25cm]{sigmoid_like.pdf}
                    \caption{Sigmoid Like}
                    \label{fig:sigmoid_like}
                \end{figure}

                \noindent
                Note that the 0.515 and 0.505 values in equation \ref{eq:sigmoid} are set to make its output range from roughly -1 to 0 over the active range.

\chapter{End Effector Placement} \label{chap:effector-placement}
A terrain scoring and optimisation function that executes on the height map is used to optimise the anchor points of the three supporting end effectors.
This chapter covers the optimisation function and parameters used for this.


\section{Scoring}
    \subsection{Terrain Proximity}
    \subsection{Slope}
    \subsection{Height Delta}

\section{Placement Optimisation}
    \subsection{Cost Function}
    \subsection{Optimisation Function}

\chapter{Hardware Implementation} \label{chap:hardware}
This chapter describes the process of implementing the system built in previous chapters on the physical robot using \ac{ros} and parallel computation.
% \ac{ros} is used 
% for the hardware implementation, first the \ac{ros} data types are defined, after which the control and analysis system running on the base station
% is described. Next the systems running on board the robot are described.

\section{ROS Nodes}
There are various ros nodes spread out across the base station, the Jetson Nano and the Teensy \ac{mcu}, figure \ref{fig:nodes} provides and overview
of the nodes and how they communicate with each other. Section \ref{sec:base_ros} and \ref{sec:on_board_ros} go into detail on the publishers and subscribers
on board the robot and on the base station. Figure \ref{fig:nodes} provides and overview of the \ac{ros} nodes and communication used to construct the system.
Further detail pertaining to communication using \ac{ros} is provided in section \ref{sec:ros_comms}.

\captionsetup[figure]{oneside,margin={0cm,0cm}}
\begin{figure}[h]
    \centering
    \includegraphics{Diagrams-Nodes.drawio.pdf}
    \caption{ROS nodes and communication.}
    \label{fig:nodes}
\end{figure}

\newpage
\section{ROS communication} \label{sec:ros_comms}
    This section provides a detailed description of the \ac{ros} communication scheme used in the system as shown in figure \ref{fig:nodes}.
    \subsection{Base Station} \label{sec:base_ros}
        The base station wirelessly communicates with the robot to send commands and receive data.
        For a description of the base station publishers see table \ref{tab:base_pubs} and for its subscribers see \ref{tab:base_subs}. Table \ref{tab:data_types} describes
        data types used.
        \begin{table}[h]
            \centering
            \begin{tabularx}{\textwidth}{| l | l | X | l |}
                \hline
                \multicolumn{4}{|c|}{\textbf{Base Station Publishers}} \\ \hline
                \textbf{Name} & \textbf{Data Type} & \textbf{Description} & \textbf{Frequency} \\ \hline
                % walk\_dir & Description & Data Type & Frequency \\ \hline
                command\_data & HexapodCommands & Various robot command parameters. & On change \\ \hline
                mode & Int32 & Specifies the oporationg mode of the robot. & On change. \\ \hline
            \end{tabularx}
            \caption{Base station publishers}
            \label{tab:base_pubs}
        \end{table}
        
        \noindent
        The robot currently only has two operating modes, torque cutoff mode and walking mode. The torque cutoff mode is the initial mode the robot is in,
        while in this mode the leg servos disable all torque control, thus entering a relaxed state. In walking mode the robot walks in the commanded direction
        whilst optimising its foot positions according to the terrain. If the robot encounters a piece of terrain for which no optimisation can be found the
        human controller will have to adjust the walking direction from the base station.

        \begin{table}[h]
            \centering
            \begin{tabularx}{\textwidth}{| l | l | X |}
                \hline
                \multicolumn{3}{|c|}{\textbf{Base Station Subscribers}} \\ \hline
                \textbf{Name} & \textbf{Data Type} & \textbf{Description} \\ \hline
                % walk\_dir & Description & Data Type & Frequency \\ \hline
                rgb\_data & Image & The processed color image from the robot. \\ \hline
                d\_data & Image & The processed color depth from the robot. \\ \hline
                hmap\_data & Image & The heightmap generated on the robot. \\ \hline
                LOGDATA & String & General logs from the robot. \\ \hline
            \end{tabularx}
            \caption{Base station subscribers}
            \label{tab:base_subs}
        \end{table}
        
        \noindent
        The only subscribers present on the base station are the processed camera images, heightmap and logs. These are all used to provide a interface from where
        the operator can control the robot.

    \subsection{On Board} \label{sec:on_board_ros}
        The hexapod has two computational units on board, first the Jetson Nano, which handles all high level operations, including heightmap generation and scoring,
        foot optimisation, maintain a walking gait and localisation using ORB-SLAM3. Secondly a Teensy2.0 \ac{mcu} handles low level operations, including interpolating feet movement paths
        and servo control. Table \ref{tab:jetson_pubs} to \ref{tab:teensy_subs} describe the \ac{ros} publishers and subscribers present on these two computational units.
        \begin{table}[h]
            \centering
            \begin{tabularx}{\textwidth}{| l | l | X | l |}
                \hline
                \multicolumn{4}{|c|}{\textbf{Jetson Publishers}} \\ \hline
                \textbf{Name} & \textbf{Data Type} & \textbf{Description} & \textbf{Frequency} \\ \hline
                % walk\_dir & Description & Data Type & Frequency \\ \hline
                effector\_targets & EffectorTargets & Data indicating which feet to move where, and what type of interpolation to use. & On change\\ \hline
                rgb\_data & Image & The processed color image from the \ac{rgbd} camera. & 15Hz. \\ \hline
                d\_data & Image & The processed depth image from the \ac{rgbd} camera. & 15Hz. \\ \hline
                hmap\_data & Image & The heightmap generated on the robot. & 15Hz. \\ \hline
                position & Vector3 & The localised position of the robot & 15Hz \\ \hline
                rotation & Quat & The localised rotation of the robot & 15Hz \\ \hline
            \end{tabularx}
            \caption{Jetson publishers}
            \label{tab:jetson_pubs}
        \end{table}

        \noindent
        The camera data, heightmap data, position and rotation are published for display at the base station.
        While the effector targets are published for use on the Teensy to move the robot's feet to the optimised positions.
        \begin{table}[h]
            \centering
            \begin{tabularx}{\textwidth}{| l | l | X |}
                \hline
                \multicolumn{3}{|c|}{\textbf{Jetson Subscribers}} \\ \hline
                \textbf{Name}  & \textbf{Data Type} & \textbf{Description} \\ \hline
                % walk\_dir & Description & Data Type & Frequency \\ \hline
                command\_data & HexapodCommands & Commands from the base station. \\ \hline
                color/image\_raw & Image & Raw color image from the \ac{rgbd} camera. \\ \hline
                aligned\_depth\_to\_color/image\_raw & Image & Raw depth image from the \ac{rgbd} camera. \\ \hline
            \end{tabularx}
            \caption{Jetson subscribers}
            \label{tab:jetson_subs}
        \end{table}

        As can be seen from table \ref{tab:jetson_subs} the only subscribers required on the Jetson is the raw camera feed, 
        for constructing the heightmap, and the commands from the base station.

        Table \ref{tab:teensy_pubs} show that the Teensy publishes the current feet positions these are the positions calculated through \ac{fk}. Additionally
        log data is also published for use on the base station.
        \begin{table}[h]
            \begin{tabularx}{\textwidth}{| l | l | X | l |}
                \hline
                \multicolumn{4}{|c|}{\textbf{Teensy Publishers}} \\ \hline
                \textbf{Name} & \textbf{Data Type} & \textbf{Description} & \textbf{Frequency} \\ \hline
                LOGDATA & String & General logs. & 10Hz \\ \hline
                effector\_current\_position & Eigen::Vector3d & Current feet positions. & 10Hz \\ \hline
            \end{tabularx}
            \caption{Teensy publishers}
            \label{tab:teensy_pubs}
        \end{table}
        \begin{table}[h]
            \centering
            \begin{tabularx}{\textwidth}{| l | l | X |}
                \hline
                \multicolumn{3}{|c|}{\textbf{Teensy subscribers}} \\ \hline
                \textbf{Name} & \textbf{Data Type} & \textbf{Description} \\ \hline
                % walk\_dir & Description & Data Type & Frequency \\ \hline
                command\_data & HexapodCommands & The color image from the \ac{rgbd} camera. \\ \hline
                effector\_targets & EffectorTargets & Data indicating which feet to move where, and what type of interpolation to use.\\ \hline
                mode & Int32 & Receives mode data from the base station \\ \hline
            \end{tabularx}
            \caption{Teensy subscribers}
            \label{tab:teensy_subs}
        \end{table}

        Lastly, from table \ref{tab:teensy_subs} it can be seen that the Teensy subscribes to the command data, effector targets and mode.
        The walking speed component from the command data is used to set the rotational rate of the servos, as discussed in section \ref{sec:ang_rate}.
        The effector targets are used to interpolate a curve for the feet to move along, as described in section \ref{sec:arc_generation}. The mode is required
        as some modes could integrate directly with the servo control, namely the torque cutoff mode.

    \newpage
    \subsection{ROS Data Types}
        Various custom \ac{ros} data types are defined to assist with communication, these data types are described in table \ref{tab:data_types}.
        \begin{table}[h]
            \centering
            \begin{tabularx}{\textwidth}{| l | p{\widthof{float32\([2]\) walk\_dir}} | X |}
                \hline
                \textbf{Name} & \textbf{Type Definition} & \textbf{Description} \\ \hline
                % walk\_dir & Description & Data Type & Frequency \\ \hline
                Vector3 & float\([3]\) data. & A vector in 3D space.  \\
                \hline
                EffectorTargets & Vector3\([6]\) targets \newline 
                                bool\([6]\) swinging. & Data describing the targets of the robot's feet and which feet are swinging. \\
                \hline
                HexapodCommands & float32\([2]\) walk\_dir \newline
                                float32 speed \newline
                                float32 height & Data packet containing various command parameters for the robot. \\
                \hline
            \end{tabularx}
            \caption{\ac{ros} data type descriptions}
            \label{tab:data_types}
        \end{table}

\newpage
\section{GPU Compute Pipeline}
    As the heightmap generation and heightmap scoring systems are essentially image manipulation processes, parallelisation of the algorithms
    is a very efficient way to increase computational speeds, thus, these algorithms are run in parallel on the Jetson nano's \ac{gpu} using OpenGL
    compute shaders. This section describes the compute pipeline used to build and score the heightmap. The \ac{gpu} pipeline can be seen in
    figure \ref{fig:compute_pipe}.
    \captionsetup[figure]{oneside,margin={0cm,0cm}}
    \begin{figure}[h]
        \centering
        \includegraphics{Diagrams-ComputePipeline.drawio.pdf}
        \caption{Compute pipeline.}
        \label{fig:compute_pipe}
    \end{figure}

    \noindent
    As seen from figure \ref{fig:compute_pipe} the \ac{gpu} pipeline is relatively simple, being comprised of only two stages, the heightmap generation
    stage and the heightmap processing stage. The heightmap generation stage executes for each pixel on the depth image, constructing the heightmap as 
    described in chapter \ref{chap:mapping}.
    
    After the heightmap has been generated, the heightmap processing stage operates over the heightmap buffer. This stage has two tasks, to erase the
    old height data as the robot walks around, and to generate the score map, as described in section \ref{sec:scores}.
    
    \subsection{A note on \ac{gpu} architecture}
    A process on a \ac{gpu} operates fully in parallel and the \ac{gpu} is highly optimised for parallelisation, thus there is a very specific
    execution structure that a \ac{gpu} process abide by to perform at maximum efficiency, or to even function at all. This execution structure is
    as follows.
    
    When writing compute code a 3D size is specified, the localgroup size, \(\bm{N}_{l} = [X_l,Y_{l},Z_{l}]\), next when the \ac{cpu} dispatches
    a compute task, the workgroup count, is fed as parameter, \(\bm{n}_{w} = [x_{w},y_{w},z_{w}]\). The \ac{gpu} then initialises
    \(\bm{n}_w\) workgroups, and each workgroup initialises \(\bm{N}_l\) threads. These threads are grouped into warps, or waves, which, depending on the architectures processor count,
    can be either 32 or 64 threads. To ensure maximum efficiency it is important that \(\bm{N}_l\) is divisible by the warp size.
    Originally Nvidia \ac{gpu}s utilised a warp size of 32 and AMD 64, however with AMDs latest RDNA architecture, the warp size could be either 32 or 64.
    This system assumes a warp size of 32.
    
    For the heightmap generation stage \(\bm{N}_{l} = [32,32,0]\) for a total of 1024 threads per workgroup, or in other words, 32 warps per workgroup.
    As for the heightmap processing stage, \(\bm{N}_{l} = [32,32,32]\), meaning 32768 threads, or 1024 warps per workgroup.
    As such it is important that camera images, the heightmap and the score map are of a size divisible by 32.
    % A basic diagram of this execution scheme can be seen in figure \ref{fig:gpu_scheme}.
    % \begin{figure}[h]
    %     \centering
    %     \includegraphics{Diagrams-ComputePipeline.drawio.pdf}
    %     \caption{GPU execution scheme.}
    %     \label{fig:gpu_scheme}
    % \end{figure}

    % \noindent
    While threads can directly communicate with each other within the same workgroup, direct communication across workgroups is
    impossible. If cross workgroup communication is desired, this must be performed using \ac{gpu} buffers, which are slower to access. Thus processes are
    designed to operate fully independently from each other. Of course, data transfer between the \ac{cpu} and the \ac{gpu} is orders of magnitudes
    slower than accessing local buffers, as such, all \ac{cpu}, \ac{gpu} communication is kept to a minimum.
    
    Finally it should be noted that while it is possible to vary \(\bm{s_w}\) with each execution cycle, \(\bm{S_l}\) is a constant specified at compile time,
    and as such the number of threads per workgroup cannot be altered during operation.

\bigskip
\bigskip
\hrule
\smallbreak
\hrule
\chapter{Final Walking Test}
    The final walking tests are comprised of three different terrains: a simple flat plane as a baseline, then a staircase to demonstrate simple foot and body height adjustment, and finally a uneven organic like surface.
    
    % When presenting testing data, a snapshot of the 3D simulation environment will be shown, with the body and foot paths traced. Additionally key data points
    % are also graphed, however only the height of a single foot will be graphed, this is to maintain clarity.

    \section{Flat Terrain}
    The flat plane test aims to quantify nominal body height oscillations and to provide a baseline to compare subsequent test to. The test was performed with a nominal stride length of 15cm, while the flow function parameters \(Ch\) and \(q\) were set to 2.0 and 14.0 respectively. This equates to a desired step height roughly equal to true stride lenght, which is dependant on the terrain. The target body height was set to \(9cm\) above the terrain. The only commands given to the robot was to walk forwards at a consant speed. Figure \ref{fig:plane_test} shows screenshots of the simulation test being performed in \ac{mujoco}. A video of the simulation can be seen \hyperlink{}{here}.

    \newpage
    \begin{figure}[h]
        \centering
        \includegraphics[width=.7\textwidth]{WalkTestPlane.png}
        \includegraphics[width=.7\textwidth]{WalkTestPlaneTop.png}
        \caption{Flat plane walk test}
        \label{fig:plane_test}
    \end{figure}
    \begin{figure}[h]
        \centering
        \includegraphics{data_plane.pdf}
        \caption{Body height (top). Body tilt (center). Position of one foot (bottom)}
        \label{fig:plane_test_data}
    \end{figure}
    
    \noindent
    The simulation results are show in Figure \ref{fig:plane_test_data}. It can be seen that the stride length is roughly equal to the set stride of \(15cm\). The step height is roughly \(10cm\) which is lower than 
    
    It can be seen that there is a constant error of about 1cm in the body height of the robot. The reason for this is that there is no body height feedback control, as such, errors in the servos accumulate to result in the body height error. This can be easily solved by adding a constant offset or by adding feedback control for the body height. It is also clear that as the robot walks there are some oscillations in the body height. This is to be expected as the servos are not modeled as having infinite torque, thus, the robot will sag when three legs are lifted off the ground, and rebound once those legs are placed on the floor again. Similarly to the body height error, adding feedback control on the body height would reduce these oscillations. The amplitude of the oscillations are lower than the \(10\%\) of the desired body height, which is acceptable. 

    Next, it can be seen that the tilt of the body also exhibit small oscillations, in both roll and pitch. However, the amplitudes of these oscillations remain below 0.5 degrees. Finally, the height of one of the legs are shown, from this it is clear that the robot takes uniform, stable steps.

    \newpage
    \section{Staircase}
    The staircase test demonstrates the ability of the robot to adjust the height of its feet in order to maintain a level body,
    additionally the ability of the robot to automatically adjust its height relative to the floor is also demonstrated. A 3D image
    of this test can be seen in figure \ref{fig:stairs_test} and the data plots in figure \ref{fig:stairs_test_data}.
    \begin{figure}[h]
        \centering
        \includegraphics[width=.8\textwidth]{WalkTestStairts.png}
        \includegraphics[width=.8\textwidth]{WalkTestStairtsTop.png}
        \caption{Stairs walk test}
        \label{fig:stairs_test}
    \end{figure}
    \begin{figure}[h]
        \centering
        \includegraphics{data_stairs.pdf}
        \caption{Body height (top). Body tilt (center). Position of one foot (bottom)}
        \label{fig:stairs_test_data}
    \end{figure}

    \noindent
    Similarly to the flat plane test, as expected, the constant body height error is still present.
    However, from this test it can be seen in figure \ref{fig:stairs_test_data} that the robot does increase its height to maintain a constant
    height above the terrain. It is clear that the body height is not increased in the four discrete steps as the stairs do, rather the
    body height is increased more smoothly using many smaller steps. This is thanks to the system described in section \ref{sec:height_adjust}.

    Body tilt, while more prominent than in the flat plane test, is still quite low as it largely stays below \(1.0^\circ\), with intermittent jumps approaching \(2.0^\circ\).
    This is still within the specifications, and as previously stated could easily be improved by incorporating accelerometer based feedback control on the foot height.

    \noindent
    Finally, the foot height plot clearly shows how one of the feet of the robot adjusts its height and step arc based on the terrain. Otherwise the foot arc looks very similar to that of the
    flat plane test, which is optimal.
    
    \newpage
    \section{Organic}
    The organic test aims to show how the robot can place feet on an appropriate spot on the terrain, while maintaining level and a certain height. Figure \ref{fig:org_test} shows the
    3D view of the test.
    \captionsetup[figure]{oneside,margin={0.9cm,0cm}}
    \begin{figure}[h]
        \centering
        \includegraphics[width=.8\textwidth]{WalkTestOrg.png}
        \includegraphics[width=.8\textwidth]{WalkTestOrgTop.png}
        \caption{Organic walk test.}
        \label{fig:org_test}
    \end{figure}
    \begin{figure}[h]
        \centering
        \includegraphics{cobble_test_feet.pdf}
        \caption{Feet top view (Top). Feet side view (Bottom)}
        \label{fig:org_test_data}
    \end{figure}
    \begin{figure}[h]
        \centering
        \includegraphics{cobble_test_body.pdf}
        \caption{Body height (Top). Body tilt (Bottom)}
        \label{fig:org_test_data}
    \end{figure}

    \noindent
    From figure \ref{fig:org_test_data} it can be seen that the body height error seen in the flat plane test
    is still present and more severe. But the body does stay within a reasonable margin from the desired height.

    For most part the body tilt is similar to that of the stairs test, however more frequent and more sever spikes are present.
    This is due, in part, to the more erratic mismatch between the body height and the requested body heigh. The horizontal adjustments made to 
    the foot end positions cause the swinging feet to end their steps at different times, this is the primary cause of the tilt spikes, as when one
    foot is placed on the floor before the other two, the robot is tilted slightly. This, however, would not occur if there was no, or little, 
    error in the body height.

    Finally, when looking at the foot position plot in figure \ref{fig:org_test_data}, please note that the plot is a projection into the X-Z plane, and thus any
    adjustment made in the Y-axis is not shown, the foot shown has been chosen to include minimum Y-axis variation. 
    Thus, please also see the 3D view in figure \ref{fig:org_test}. From these two figures it is clear that the foot end positions are, in addition to height,
    adjusted in the horizontal plane, the step arcs are appropriately adjusted and remain relatively smooth, similar to the flat plane test.


% \bigskip
% \bigskip
% \hrule
% \smallbreak
% \hrule


\chapter{Conclusions}
    \section{Summary and Conclusions}
        This thesis presented the development of a foot placement planning system for a hexapod robot to enable it to traverse uneven terrain containing height changes, sloped surfaces, and edges.
        A simulation model of the existing physical hexapod robot, and its environment, was created in \ac{mujoco}, an advanced physics engine with superior contact physics modelling.
        A real time local dense-mapping system was developed by using depth images obtained from an \ac{rgbd} camera mounted on the robot. The mapping  system also utilised pose estimations from ORB-SLAM3, a stereoscopic visual SLAM algorithm. ORB-SLAM3 was implemented using a third party library.
        A walkability scoring system was developed to further process the heightmap generated by the mapping system, resulting in a walkability map. The walkability map indicates where on the terrain the robot can and can not safely step.
        Next, a baseline motion control system that enables the robot to walk on flat terrain was developed. The baseline system was then further modified with a algorithm that adjusts the nominal foot positions from the baseline motion control system based on the score map. The resulting new foot positions are then adjusted both in the horizontal and vertical plane to a position on the terrain that is safe to step on, all while the robot maintains a level body. The robot's floor height reference was also updated based on the terrain map, this allowed the robot to adjust its body height appropriately to the average terrain height. Furthermore, a new foot trajectory generation system based on a flow function was developed to produce a trajectory based on the horizontal and vertical distance to the foot's destination.

        The mapping, scoring, and optimal foot placement systems were tested in simulation and on practical RGB-D images recorded by the physical hexapod robot. Simulation tests were performed to demonstrate that the integrated system can successfully control the hexapod robot to walk on flat terrain, staircase terrain, and uneven terrain consisting of "cobblestones"  with grooves between them. The simulation results show that the hexapod is able to traverse both flat and uneven terrain, is able to adjust both its horizontal and vertical foot placements in response to the terrain.

    \section{Future Work}
        Below is listed possible future tests, improvements, and additions to the system.

        \begin{enumerate}
            \item Although the mapping and scoring was implemented and practically tested on the physical hexapod, the motion control system was only tested in simulation on uneven terrain. Future work should include practical tests with the physical hexapod traversing uneven terrain such as steps and cobblestones.            
            \item The variations in the height estimate received from the ORB-SLAM3 system caused artifacts in the heightmap and the score map. Methods to improve the height estimate should be investigated.
            \item The foot arcs are currently only adjusted for the new foot placements, but are not adjusted to avoid obstacles in the terrain. Future work could look at detected obstacles close to the legs and adjusting he arcs to avoid collisions.
            \item The hexapod currently "freezes" if it cannot find suitable alternate foot placement positions for the given terrain, and the human operator must then manually change the commanded direction of motion. A short-term path planner should be investigated that can autonomously change the hexapod's planned path to circumvent the terrain that cannot be crossed.
            \item The mapping and scoring system currently does not identify hazardous terrain such as pools of water or loose earth. Terrain classification and image segmentation algorithms could be investigated to add a "terrain type" score to to the walkability score.

            
        \end{enumerate}

\appendix%------------------------------------------------------------
\chapter{Reference Frame Transforms} \label{app:transforms}
    Reference frame transforms are defined by the shorthand \(\bm{x}\inrefframe{A}\transframe{A}{B}\), meaning vector \(\bm{x}\inrefframe{A}\) is transformed
    from reference frame \(\refframe{A}\) to \(\refframe{B}\). Reference frames used are the camera frame, \(\refframe{C}\),
    the map frame, \(\refframe{M}\), the local frame, \(\refframe{L}\), and the global frame, \(\refframe{G}\). Transforms used are
    descried in this appendix.

    \section{Camera to Map}
        Transforming a vector from camera,\ to map space requires rotating the vector by the camera quaternion, adding
        te camera's global position to the vector, scaling the vector into the map by multiplying with the map scaling
        factor, and finally modulating the vector to the confines of the map. As shown in equation \ref{eq:camera_to_map}.
        \begin{align} \label{eq:camera_to_map}
        \begin{split}
            \bm{x}\inrefframe{M} &= \bm{x}\inrefframe{C}\transframe{C}{M}\\ 
            &= (\bm{q}_\text{cam} \cdot \bm{x}\inrefframe{C} \cdot \bm{q}_\text{cam}^{-1}S + \bm{p_\text{rob}}\inrefframe{M}) \mod N
        \end{split}
        \end{align}
        where \(\bm{x}\inrefframe{C}\) is the vector in camera space, \(\bm{q}_\text{cam}\) is the camera quaternion that rotates \(\bm{x}\inrefframe{C}\) 
        into world space, with its \(z\) axis pointing upwards, \(\bm{p_\text{rob}}\inrefframe{M}\) is the coordinate vector of the robot in map space, and \(S\) is the scaling factor
        used to relate heightmap cells to cm. The relationship between \(N, S\) and \(E\) are characterised by equation \ref{eq:map_scaling}.
        \begin{equation} \label{eq:map_scaling}
            N = E \cdot S
        \end{equation}
        where \(E\times E\) is the size of the \(N\times N\) heightmap buffer in cm.
    
    \newpage
    \section{World to Map}
        The world to map transform is similar to camera to map tansform, except that the rotation is not required, as the map and world frames are already aligned.
        This is shown in equation \ref{eq:world_to_map}.
        \begin{align}\label{eq:world_to_map}
        \begin{split}
            \bm{x}\inrefframe{M} &= \bm{x}\inrefframe{W} \transframe{W}{M} \\
            &= \bm{x}\inrefframe{W}S \mod{N}
        \end{split}
        \end{align}
        where \(S\) is the scaling factor used to relate heightmap cells to cm, and the heightmap buffer is of size \(N\times N\).
    \section{Local to Map}
        Local to map space is very similar to equation \ref{eq:world_to_map}, with the exception of using the body quaternion instead of the camera quaterion. This can be seen in
        equation \ref{eq:local_to_map}.
        \begin{equation} \label{eq:local_to_map}
            \bm{x}\inrefframe{M} = (\bm{q_\text{bod}} \cdot \bm{x}\inrefframe{L} \cdot \bm{q_\text{bod}}^{-1}S + \bm{p_\text{rob}}\inrefframe{M}) \mod N
        \end{equation}
        where \(\bm{x}\inrefframe{L}\) is the vector in local space, \(\bm{q}_\text{bod}\) is the robot body's quaternion that rotates \(\bm{x}\inrefframe{L}\) 
        into map space, with its \(z\) axis pointing upwards, \(\bm{p_\text{rob}}\inrefframe{M}\) is the coordinate vector of the robot in map space, and \(S\) is the scaling factor
        used to relate heightmap cells to cm.

    \section{Map to Local}
        Due to the circular nature of the map buffer, translating from map to local space is more challenging. First an intermediate position, \(\bm{x_\text{temp}}\), is defined as a helper,
        \begin{equation} \label{eq:map_to_local_help}
            \bm{x_\text{temp}} = \bm{p_\text{rob}}\inrefframe{M}\frac{1}{S}
        \end{equation}
        where \(\bm{x_\text{temp}}\) is the map coordinates with the inverse map scaling applied. Now \(\bm{x_\text{temp}}\) must be checked to see if either of its 
        coordinates exceed half the map extents, if so, then it must be adjusted such that both coordinates fall within \(\frac{1}{2}E\).
        This is expressed in equation \ref{eq:map_to_local_adj}
        \begin{equation} \label{eq:map_to_local_adj}
            \bm{x_\text{temp}} \Leftarrow 
            \begin{cases}
                \bm{x_\text{temp}} & \bm{x_\text{temp}} \leq \frac{1}{2}E \\
                \bm{x_\text{temp}} - E(\sgn \bm{x_\text{temp}}) & \bm{x_\text{temp}} > \frac{1}{2}E
            \end{cases}
        \end{equation}
        where \(\sgn\) is the signum function, which return either 1 or -1 depending on the sign of the operand. 
        \textbf{Note} that equation \ref{eq:map_to_local_adj} is applied to both coordinates in the vector \(\bm{x_\text{temp}}\) separately, in pursuit of readability this is not explicitly stated.

        Now that \(\bm{x_\text{temp}}\) has been corrected for the circular nature of the map buffer, only the rotating into the local space remains. As show in equation \ref{eq:map_to_local_rot}.
        \begin{align} \label{eq:map_to_local_rot}
        \begin{split}
            \bm{x}\inrefframe{L} &= \bm{x}\inrefframe{M} \transframe{M}{L} \\
            &= \bm{q_\text{bod}}^{-1} \cdot \bm{x_\text{temp}} \cdot \bm{q_\text{bod}}
        \end{split}
        \end{align}
        where \(\bm{q}_\text{bod}\) is the robot body's quaternion.
\chapter{ROS Messages}\label{app:ros_comms}
    This Appendix provides a detailed account of the \ac{ros} messages used in the \ac{ros} nodes described in chapter \ref{chap:hardware}. Table \ref{tab:base_pubs} and \ref{tab:base_subs} describe the \ac{ros} publishers and subscribers present on the base station.
    \begin{table}[h]
        \centering
        \begin{tabularx}{\textwidth}{| l | l | X | l |}
            \hline
            \multicolumn{4}{|c|}{\textbf{Base Station Publishers}} \\ \hline
            \textbf{Name} & \textbf{Data Type} & \textbf{Description} & \textbf{Frequency} \\ \hline
            % walk\_dir & Description & Data Type & Frequency \\ \hline
            command\_data & HexapodCommands & Various robot command parameters. & On change \\ \hline
            mode & Int32 & Specifies the oporationg mode of the robot. & On change. \\ \hline
        \end{tabularx}
        \caption{Base station publishers}
        \label{tab:base_pubs}
    \end{table}
    \begin{table}[h]
        \centering
        \begin{tabularx}{\textwidth}{| l | l | X |}
            \hline
            \multicolumn{3}{|c|}{\textbf{Base Station Subscribers}} \\ \hline
            \textbf{Name} & \textbf{Data Type} & \textbf{Description} \\ \hline
            % walk\_dir & Description & Data Type & Frequency \\ \hline
            rgb\_data & Image & The processed color image from the robot. \\ \hline
            d\_data & Image & The processed color depth from the robot. \\ \hline
            hmap\_data & Image & The heightmap generated on the robot. \\ \hline
            scoremap\_data & Image & The scoremap generated on the robot. \\ \hline
            LOGDATA & String & General logs from the robot. \\ \hline
        \end{tabularx}
        \caption{Base station subscribers}
        \label{tab:base_subs}
    \end{table}

    \newpage
    \noindent
    Table \ref{tab:jetson_pubs} and \ref{tab:jetson_subs} describe the \ac{ros} publishers and subscribers present on the Jetson Nano.

    \begin{table}[h]
        \centering
        \begin{tabularx}{\textwidth}{| l | l | X | l |}
            \hline
            \multicolumn{4}{|c|}{\textbf{Jetson Publishers}} \\ \hline
            \textbf{Name} & \textbf{Data Type} & \textbf{Description} & \textbf{Frequency} \\ \hline
            % walk\_dir & Description & Data Type & Frequency \\ \hline
            effector\_targets & EffectorTargets & Data indicating which feet to move where, and what type of interpolation to use. & On change\\ \hline
            rgb\_data & Image & The processed color image from the \ac{rgbd} camera. & 15Hz. \\ \hline
            d\_data & Image & The processed depth image from the \ac{rgbd} camera. & 15Hz. \\ \hline
            hmap\_data & Image & The heightmap generated on the robot. & 15Hz. \\ \hline
            scoremap\_data & Image & The scoremap generated on the robot. & 15Hz. \\ \hline
            position & Vector3 & The localised position of the robot & 15Hz \\ \hline
            rotation & Quat & The localised rotation of the robot & 15Hz \\ \hline
        \end{tabularx}
        \caption{Jetson publishers}
        \label{tab:jetson_pubs}
    \end{table}
    % \newpage
    \begin{table}[h]
        \centering
        \begin{tabularx}{\textwidth}{| l | l | X |}
            \hline
            \multicolumn{3}{|c|}{\textbf{Jetson Subscribers}} \\ \hline
            \textbf{Name}  & \textbf{Data Type} & \textbf{Description} \\ \hline
            % walk\_dir & Description & Data Type & Frequency \\ \hline
            command\_data & HexapodCommands & Operator commands. \\ \hline
            color/image\_raw & Image & Color image from the camera. \\ \hline
            aligned\_depth\_to\_color/image\_raw & Image & Depth image from the camera. \\ \hline
        \end{tabularx}
        \caption{Jetson subscribers}
        \label{tab:jetson_subs}
    \end{table}

    \newpage
    \noindent
    Table \ref{tab:teensy_pubs} and \ref{tab:teensy_subs} describe the puvblishers and subscribers present on the Teensy \ac{mcu}.
    \begin{table}[h]
        \begin{tabularx}{\textwidth}{| l | l | X | l |}
            \hline
            \multicolumn{4}{|c|}{\textbf{Teensy Publishers}} \\ \hline
            \textbf{Name} & \textbf{Data Type} & \textbf{Description} & \textbf{Frequency} \\ \hline
            LOGDATA & String & General logs. & 10Hz \\ \hline
            effector\_current\_position & Eigen::Vector3d & Current feet positions. & 10Hz \\ \hline
        \end{tabularx}
        \caption{Teensy publishers}
        \label{tab:teensy_pubs}
    \end{table}
    \begin{table}[h]
        \centering
        \begin{tabularx}{\textwidth}{| l | l | X |}
            \hline
            \multicolumn{3}{|c|}{\textbf{Teensy subscribers}} \\ \hline
            \textbf{Name} & \textbf{Data Type} & \textbf{Description} \\ \hline
            % walk\_dir & Description & Data Type & Frequency \\ \hline
            command\_data & HexapodCommands & The color image from the \ac{rgbd} camera. \\ \hline
            effector\_targets & EffectorTargets & Data indicating which feet to move where, and what type of interpolation to use.\\ \hline
            mode & Int32 & Receives mode data from the base station \\ \hline
        \end{tabularx}
        \caption{Teensy subscribers}
        \label{tab:teensy_subs}
    \end{table}

    \newpage
    \noindent 
    Lastly, the custom \ac{ros} data types used are defined in table \ref{tab:data_types}.
    \begin{table}[h]
        \centering
        \begin{tabularx}{\textwidth}{| l | p{\widthof{float32\([2]\) walk\_dir}} | X |}
            \hline
            \textbf{Name} & \textbf{Type Definition} & \textbf{Description} \\ \hline
            % walk\_dir & Description & Data Type & Frequency \\ \hline
            Vector3 & float\([3]\) data. & A vector in 3D space.  \\
            \hline
            EffectorTargets & Vector3\([6]\) targets \newline 
                            bool\([6]\) swinging. & Data describing the targets of the robot's feet and which feet are swinging. \\
            \hline
            HexapodCommands & float32\([2]\) walk\_dir \newline
                            float32 speed \newline
                            float32 height & Data packet containing various command parameters for the robot. \\
            \hline
        \end{tabularx}
        \caption{\ac{ros} data type descriptions}
        \label{tab:data_types}
    \end{table}

\backmatter%----------------------------------------------------------
\bibliography{bib/bib-sample}
 
\end{document}   


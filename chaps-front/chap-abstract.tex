
% \begin{abstract}[english]%===================================================
\chapter*{Abstract}
In recent times great strides have been made in the field of autonomous robotics, 
especially with regards to autonomous navigation of wheeled and aerial drones.
Legged robotics however still face numerous problems before they can become practical
to use, the most egregious of these problems being balancing of the robot, and optimal foot placement.

This thesis focuses on providing a solution to the latter problem of foot placement. This is achieved by using a depth camera to, in real time, construct a localised map of the environment, and subsequently analysing  said map for optimal foot placement locations. The system is then tested using a hexapod robot, both in simulation and on a physical robot.
% \end{abstract}


\chapter*{Uittreksel}
% \begin{abstract}[afrikaans]%=================================================
    In onlangse tye is groot vordering gemaak in die gebied van outonome robotika, 
    veral met betrekking tot outonome navigasie van hommeltuie. Bebeende-robotika het egter steeds probleme om op te los voordat dit prakties gebruik kan word, die mees ernstige van hierdie probleme is balansering van die robot, en optimale voetplasing.

    Hierdie tesis fokus daarop om 'n oplossing vir die laasgenoemde probleem van voetplasing voor te stel. Dit word bereik deur 'n dieptekamera te gebruik  om 'n gelokaliseerde kaart van die omgewing te konstrueer, en daarna die kaart te ontleed vir optimale voetplasings areas. Die stelsel word dan getoets met behulp van 'n seskantige-robot, beide in simulasie en op 'n fisiese robot.
% \end{abstract}

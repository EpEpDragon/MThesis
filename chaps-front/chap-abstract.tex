
\begin{abstract}[english]%===================================================
In recent times great strides have been made in the field of autonomous robotics, 
especially with regards to autonomous navigation of wheeled and aerial drones.
Legged robotics however still face numerous problems before they can become practical
to use, the most egregious of these problems being balancing of the robot and optimal foot placement.

This thesis focuses on providing a solution to the latter problem of foot placement. This is achieved by using an 
depth camera to, in real time, construct a localised map of the environment and subsequently analysing 
said map for optimal foot placement locations. The system is then tested using a hexapod robot both in
simulation and on a physical robot.
\end{abstract}


\begin{abstract}[afrikaans]%=================================================
Om `n tand implement te vibreer is `n effektiewe manier om die trekkrag, wat
benodig word om dit deur die grond te trek, te verminder. Die graad van krag
vermindering is afhanklik van die kombinasie van werks parameters en die
grond toestand. Dus is dit nodig om die vibrerende implement te optimeer vir
verskillende omstandighede.

Numeriese modulering is meer buigsaam en goedkoper as eksperimentele
opstellings en analitiese modelle. Die Diskrete Element Metode (DEM) was
spesifiek vir korrelrige materiaal, soos grond, ontwikkel en kan gebruik word
vir die modellering van `n vibrerende implement vir die ontwerp en optimering
daarvan. Die doel was dus om die vermoë van DEM om 'n vibrerende skeurploeg
the modelleer, te evalueer, en om die oorsaak van die krag vermindering te
ondersoek.

Die DEM model was geïvalueer teen data ...
\end{abstract}

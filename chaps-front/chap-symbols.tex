\chapter{List of symbols}
% Use stb-nomenclature + siunitx

\begin{Nomencl}[2.75cm]
\NomGroup{Constants}%-----------------------------------------------
    \item[$L_0 = $] \qty{300}{mm}

\NomGroup{Variables}%-----------------------------------------------
    \item[$\mathit{Re}_\mathrm{\,D}$]
                       \UnitLine{Reynolds number (diameter)}{~}
    \item[$x$]         \UnitLine{Coordinate                }{m}
    \item[$\ddot{x}$]  \UnitLine{Acceleration              }{m/s^2}\\
    
    \item[$\theta$]    \UnitLine{Rotation angle            }{rad}
    \item[$\tau$]      \UnitLine{Moment                    }{\newton\meter}

\NomGroup{Vectors and Tensors}%-------------------------------------
    \item[$\overrightarrow{\bm{v}}$] Physical vector, see equation ...

\NomGroup{Subscripts}%----------------------------------------------
    \item[$\mathrm{a}$] Adiabatic
    \item[$a$]          Coordinate

% 
\NomGroup{Notation}
    % \setlength{\itemindent}{1cm}
    \item[\(\seq{a_0}{a_1}{a_n}\)] Defines a integer sequence from \(a_0\) to \(a_n\), with a increment of \(a_1-a_0\).
    \item[\(x\inrefframe{A}\)] Indicates that \(x\) is in reference frame \(\mathcal{A}\).
    \item[\(\transframenew{x}{\refframe{A}}{\refframe{B}}\)] Transforms \refframe{A} to reference frame \refframe{B},\footnote{See appendix \ref{app:transforms} for transform definitions.}
    \item[\(\mathcal{W}\)] Indicates the world reference frame.\footnote{\label{note1}See appendix \ref{app:frames} for reference frame definitions.}
    \item[\(\mathcal{L}\)] Indicates the local reference frame.\textsuperscript{\ref{note1}}
    \item[\(\mathcal{M}\)] Indicates the map reference frame.\textsuperscript{\ref{note1}}
    \item[\(\mathcal{C}\)] Indicates the camera reference frame.\textsuperscript{\ref{note1}}
% \NomGroup{Abreviations}%-----------------------------------------------
%     \begin{acronym}[MMIIII]
%         \item \acro{ik}[IK]{Inverse Kinematics}
%         \acro{fk}[FK]{Forward Kinematics}
%         \acro{sdf}[SDF]{Signed Distance Field}
%         \acro{mujoco}[MuJoCo]{Multi-Joint dynamics with Contact}
%         \acro{gui}[GUI]{Graphical User Interface}
%         \acro{ros}[ROS]{Robot Operating System}
%         \acro{lidar}[LiDAR]{Light Detection and Ranging}
%         \acro{rgbd}[RGB-D]{Red Green Blue Depth}
%         \acro{slam}[SLAM]{Simultaneous Localisation and Mapping}
%         \acro{imu}[IMU]{Inertial Measuring Unit}
%         \acro{rl}[RL]{Reinforcement Learing}
%         \acro{ann}[ANN]{Artificial Neural Network}
%         \acro{gps}[GPS]{Global Positioning System}
%     \end{acronym}    
\end{Nomencl}

\subsubsection*{Abbreviations}
    \hfill\begin{minipage}{\dimexpr\textwidth-\NomLblSep}
        \begin{acronym}[MMMMMMii]
            \acro{ik}[IK]{Inverse Kinematics}
            \acro{fk}[FK]{Forward Kinematics}
            \acro{sdf}[SDF]{Signed Distance Field}
            \acro{mujoco}[MuJoCo]{Multi-Joint dynamics with Contact}
            \acro{gui}[GUI]{Graphical User Interface}
            \acro{ros}[ROS]{Robot Operating System}
            \acro{lidar}[LiDAR]{Light Detection and Ranging}
            \acro{rgbd}[RGB-D]{Red Green Blue Depth}
            \acro{slam}[SLAM]{Simultaneous Localisation and Mapping}
            \acro{imu}[IMU]{Inertial Measuring Unit}
            \acro{rl}[RL]{Reinforcement Learing}
            \acro{ann}[ANN]{Artificial Neural Network}
            \acro{gps}[GPS]{Global Positioning System}
        \end{acronym}       
    \end{minipage}

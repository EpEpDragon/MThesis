\chapter{List of symbols}
% Use stb-nomenclature + siunitx

\begin{Nomencl}[1cm]
\NomGroup{Constants}%-----------------------------------------------
    \item[$L_0 = $] \qty{300}{mm}

\NomGroup{Variables}%-----------------------------------------------
    \item[$\mathit{Re}_\mathrm{\,D}$]
                       \UnitLine{Reynolds number (diameter)}{~}
    \item[$x$]         \UnitLine{Coordinate                }{m}
    \item[$\ddot{x}$]  \UnitLine{Acceleration              }{m/s^2}\\
    
    \item[$\theta$]    \UnitLine{Rotation angle            }{rad}
    \item[$\tau$]      \UnitLine{Moment                    }{\newton\meter}

\NomGroup{Vectors and Tensors}%-------------------------------------
    \item[$\overrightarrow{\bm{v}}$] Physical vector, see equation ...

\NomGroup{Subscripts}%----------------------------------------------
    \item[$\mathrm{a}$] Adiabatic
    \item[$a$]          Coordinate

\NomGroup{Notation}
    \item[\(\seq{a_0}{a_1}{a_n}\)] Defines a integer sequence from \(a_0\) to \(a_n\), with a increment of \(a_1-a_0\).
    \item[\(x\inrefframe{A}\)] Indicates that \(x\) is in reference frame \(\mathcal{A}\).
    \item[\(x\transframe{A}{B}\)] Transforms \(x\inrefframe{A}\) to reference frame \(\mathcal{B}\),\footnote{See appendix \ref{app:transforms} for transform definitions.} 
    such that the resultant point in global space stays constant, meaning, \(\left[x\inrefframe{A}\transframe{A}{B}\right]\inrefframe{B}\)
    \item[\(\mathcal{W}\)] Indicates the world reference frame.\footnote{\label{note1}See appendix \ref{app:frames} for reference frame definitions.}
    \item[\(\mathcal{L}\)] Indicates the local reference frame.\textsuperscript{\ref{note1}}
    \item[\(\mathcal{M}\)] Indicates the map reference frame.\textsuperscript{\ref{note1}}
    \item[\(\mathcal{C}\)] Indicates the camera reference frame.\textsuperscript{\ref{note1}}
    
\end{Nomencl}

% \begin{Nomencl}[1cm]
% \NomGroup{Abreviations}%-----------------------------------------------
    % \item[DEM] Discrete Element Method
    % \item[FEA] Finite Element Analysis
    \begin{acronym}[MMIIII]
    \NomGroup{Abreviations}%-----------------------------------------------
        \acro{ik}[IK]{Inverse Kinematics}
        \acro{fk}[FK]{Forward Kinematics}
        \acro{sdf}[SDF]{Signed Distance Field}
        \acro{mujoco}[MuJoCo]{Multi-Joint dynamics with Contact}
        \acro{gui}[GUI]{Graphical User Interface}
        \acro{ros}[ROS]{Robot Operating System}
        \acro{lidar}[LiDAR]{Light Detection and Ranging}
        \acro{rgbd}[RGB-D]{Red Green Blue Depth}
        \acro{slam}[SLAM]{Simultaneous Localisation and Mapping}
        \acro{imu}[IMU]{Inertial Measuring Unit}
        \acro{rl}[RL]{Reinforcement Learing}
        \acro{ann}[ANN]{Artificial Neural Network}
        \acro{gps}[GPS]{Global Positioning System}
    \end{acronym}    
% \end{Nomencl}

